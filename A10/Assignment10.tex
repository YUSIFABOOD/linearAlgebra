\documentclass[a4paper,12pt]{article}
\usepackage{graphicx, setspace, array, xcolor, mathtools, contour, tikz, amsthm, setspace, amsmath,amsfonts,amssymb, subcaption, fancyhdr, lipsum,multicol, geometry}
\usepackage{colortbl}
\geometry{a4paper, margin=1in}
\contourlength{0.5pt}
\usetikzlibrary{patterns}
\usepackage[english]{babel}
\renewcommand{\qedsymbol}{$\blacksquare$}
\definecolor{darkgreen}{rgb}{0.0, 0.5, 0.0}
\pagestyle{fancy}
\fancyhf{}
\renewcommand{\headrulewidth}{0.4pt}
\fancyhead[L]{\rightmark}
\fancyhead[R]{\thepage}
\title{Linear Algebra}
\author{Assignment 10\\ \\ Yousef A. Abood\\ \\ ID: 900248250}
\date{June 2025}
\setlength{\parindent}{0pt}
\singlespacing
\parskip=1mm
\setcounter{section}{6}
\setcounter{subsection}{0}
\onehalfspacing
\begin{document}
\maketitle
\noindent\makebox[\linewidth]{\rule{15cm}{0.4pt}}
\subsection{Introduction to Linear Transformations}
\subsubsection*{Problem 29}
\begin{itemize}
    \item [a)] To find the kernel we find the set of the $3 \times 1$ matrices which are mapped to the zero vector $\vec{\textbf{0}}_{2\times 1}$ such that $Ax=\vec{\textbf{0}}_{2\times 1}$, so the kernel is the nullspace of $A$. We solve the homogenous system $Ax=\vec{0}$ to get $N(A)$. We start by applying EROs to $A.$
    \begin{align*}A=\begin{bmatrix}
      0 & -2 & 3 \\
      4 & 0 & 11
    \end{bmatrix} \xrightarrow{{R_2}\leftrightarrow{R_1}}&
    \begin{bmatrix}
      4 & 0 & 11 \\
      0 & -2 & 3
    \end{bmatrix} \xrightarrow{{\frac{1}{4}R_1}\to{R_1}}
    \begin{bmatrix}
      1 & 0 & \frac{11}{4} \\
      0 & -2 & 3
    \end{bmatrix}\\
    \xrightarrow{{\frac{-1}{2}R_2}\to{R_2}}&
    \begin{bmatrix}
      1 & 0 & \frac{11}{4} \\
      0 & 1 & \frac{-3}{2}
    \end{bmatrix}
\end{align*}
We see we get the system \begin{align*}x+\frac{11}{4}z=0 \\ y-\frac{3}{2}z=0\end{align*}, the non-pivot column of the ERO matrix is the $3^{rd}$ column and $z$ is the free variable. So we let $z=t, t\in \mathbb{R}$, and we get $x=-\frac{11}{4}t, y=\frac{3}{2}t.$
\begin{align*}
    \text{So } N(A)=
    \left\{\begin{bmatrix}
    -\frac{11}{4}t\\
    \frac{3}{2}t\\
    t
    \end{bmatrix}: t \in \mathbb{R}\right\} =  \left\{t\begin{bmatrix}
    -\frac{11}{4}\\
    \frac{3}{2}\\
    1
    \end{bmatrix}: t \in \mathbb{R}\right\} = Span\left(\begin{bmatrix}
    -\frac{11}{4}\\
    \frac{3}{2}\\
    1
    \end{bmatrix}\right)
\end{align*}
    \item [b)] $nullity(T)=nullity(A)=1$, the number of the non-pivot columns.
    \item [c)] We use the fact that $range(T)=C(A)$. To find the basis of the column space, we find the columns in $A$ that correspond to the pivot column in the ERO matrix:
\begin{align*}
    \text{basis of }C(A)=
    \left\{
        \begin{bmatrix}
            0\\
            4
        \end{bmatrix}, \begin{bmatrix}
            -2\\
            0
        \end{bmatrix}
    \right\}, C(A)=span \left\{
        \begin{bmatrix}
            0\\
            4
        \end{bmatrix}, \begin{bmatrix}
            -2\\
            0
        \end{bmatrix}
    \right\}
\end{align*}
    \item [d)] We use the fact that $rank(T)+nullity(T)=rank(A)+nullity(A)=n$, where $n$ is the number of columns of $A$. So $rank(T)=3-nullity(T)=3-1=2.$
\end{itemize}
\subsubsection*{Problem 40}
Observe that $T(x,y,z)=(x,y,0)$ can be written in the form $A\vec{x}=\vec{b}:$
\begin{align*}
    \begin{bmatrix}
      1 & 0 & 0 \\
      0 & 1 & 0 \\
      0 & 0 & 0
    \end{bmatrix} \begin{bmatrix}
        x\\
        y\\
        z
    \end{bmatrix} = \begin{bmatrix}
        x\\
        y\\
        0
    \end{bmatrix}.
\end{align*}
$A$ is clearly in REF, and has two pivot columns which are the $1^{st}, 2^{nd}$ columns, so $rank(A)=2.$ Using the fact that $rank(T)+nullity(T)=rank(A)+nullity(A)=n$ we deduce that $nullity(T)=3-2=1.$
\\
To find the kernel, we solve the system $A\vec{x}=\vec{b}$ when $\vec{b}=\vec{\textbf{0}_{3\times 3}}$. We see that $A$ is already in REF, so we don't need to apply the Gaussian elimination method. We get the system : $x=0, y=0$ and $z$ is a free varible since it corresponds to the third column which is a non-pivot column. So \begin{align*}
    ker(T)=N(A)= 
     \left\{
        \begin{bmatrix}
            0\\
            0\\
            t
        \end{bmatrix}: t \in \mathbb{R}
    \right\}
\end{align*}
Clearly, the geometric representation of the kernel is the set of all points in the $z-axis$ in $\mathbb{R}^3.$

To find the range, we use that $range(T)=C(A).$ Since $A$ is in REF, we pick the columns in $A$ that are pivot columns to get the basis of the column space. 
\begin{align*}
    C(A)= span\left\{
        \begin{bmatrix}
            1\\
            0\\
            0
        \end{bmatrix},\begin{bmatrix}
            0\\
            1\\
            0
        \end{bmatrix}
    \right\} =range(T)
\end{align*}
Which is interpretted geometrically as the set of every point in $xy-plane$ in $\mathbb{R}^3.$
\setcounter{subsection}{2}
\subsection{Matrices for Linear Transformations}
\subsubsection*{Problem 9}
We can see that this linear transformation is from $\mathbb{R}^2 \to \mathbb{R}^3$, so firstly, we find the images of the standard basis of $\mathbb{R}^2$, namely, $T(e_1),T(e_2).$
\begin{align*}
    T\left(
        \begin{bmatrix}
            1\\
            0
        \end{bmatrix}
    \right)= \begin{bmatrix}
        1\\
        2\\
        0
    \end{bmatrix}, T\left(
        \begin{bmatrix}
            0\\
            1
        \end{bmatrix}
    \right)= \begin{bmatrix}
        -3\\
        1\\
        1
    \end{bmatrix}.
\end{align*}
Hence, the standard matrix is  
\[A=\begin{bmatrix}
    1&-3\\
    2&1\\
    0&1
\end{bmatrix}\]
and the image of $\vec{v}=T(\vec{v})=A\vec{v}=$
\begin{align*}
    \begin{bmatrix}
    1&-3\\
    2&1\\
    0&1
\end{bmatrix} \begin{bmatrix}
    -2\\
    4
\end{bmatrix} = \begin{bmatrix}
    -14\\
    0\\
    4
\end{bmatrix}
\end{align*}
\subsubsection*{Probme 40}
\begin{itemize}
    \item [a)] Since the domain is $R^4$, we find the images of the standard basis, \\namely, $T(e_1),T(e_2),T(e_3),T(e_4)$
\begin{align*}
    T\left( 
        \begin{bmatrix}
            1\\
            0\\
            0\\
            0
        \end{bmatrix}
    \right) = \begin{bmatrix}
        1\\
        -1
    \end{bmatrix}, T\left( 
        \begin{bmatrix}
            0\\
            1\\
            0\\
            0
        \end{bmatrix}
    \right) = \begin{bmatrix}
        1\\
        0
    \end{bmatrix}, T\left( 
        \begin{bmatrix}
            0\\
            0\\
            1\\
            0
        \end{bmatrix}
    \right) = \begin{bmatrix}
        1\\
        0
    \end{bmatrix}, T\left( 
        \begin{bmatrix}
            0\\
            0\\
            0\\
            1
        \end{bmatrix}
    \right) = \begin{bmatrix}
        1\\
        1
    \end{bmatrix}
\end{align*}
Thus, the standard matrix is
\[A=
\begin{bmatrix}
    1&1&1&1\\
    -1&0&0&1
\end{bmatrix}
\]
and the image of $\vec{v}=$
\begin{align*}
    \begin{bmatrix}
    1&1&1&1\\
    -1&0&0&1
\end{bmatrix}
\begin{bmatrix}
            4\\
            -3\\
            1\\
            1
        \end{bmatrix}= \begin{bmatrix}
            3\\
            -3
        \end{bmatrix}
\end{align*} 
    \item[b)] We see
    \begin{align*}
        T\left(
            \begin{bmatrix}
                1\\
                0\\
                0\\
                1
            \end{bmatrix}
        \right) = \begin{bmatrix}
            2\\
            0
        \end{bmatrix} = 0 \begin{bmatrix}
          1 \\
          1 
        \end{bmatrix} + (1)\begin{bmatrix}
            2\\
            0
        \end{bmatrix}.\text{ So}, [T(1,0,0,1)]_B= \begin{bmatrix}
            0\\
            1
        \end{bmatrix}.\\
        T\left(
            \begin{bmatrix}
                0\\
                1\\
                0\\
                1
            \end{bmatrix}
        \right) = \begin{bmatrix}
            2\\
            1
        \end{bmatrix} = 1 \begin{bmatrix}
          1 \\
          1 
        \end{bmatrix} + \frac{1}{2}\begin{bmatrix}
            2\\
            0
        \end{bmatrix}.\text{ So}, [T(0,1,0,1)]_B= \begin{bmatrix}
            1\\
            \frac{1}{2}
        \end{bmatrix}.\\
        T\left(
            \begin{bmatrix}
                1\\
                0\\
                1\\
                0
            \end{bmatrix}
        \right) = \begin{bmatrix}
            2\\
            -1
        \end{bmatrix} = -1 \begin{bmatrix}
          1 \\
          1 
        \end{bmatrix} + \frac{3}{2}\begin{bmatrix}
            2\\
            0
        \end{bmatrix}.\text{ So}, [T(0,1,0,1)]_B= \begin{bmatrix}
            -1\\
            \frac{3}{2}
        \end{bmatrix}.\\
         T\left(
            \begin{bmatrix}
                1\\
                1\\
                0\\
                0
            \end{bmatrix}
        \right) = \begin{bmatrix}
            2\\
            -1
        \end{bmatrix} = -1 \begin{bmatrix}
          1 \\
          1 
        \end{bmatrix} + \frac{3}{2}\begin{bmatrix}
            2\\
            0
        \end{bmatrix}.\text{ So}, [T(0,1,0,1)]_B= \begin{bmatrix}
            -1\\
            \frac{3}{2}
        \end{bmatrix}.\\      
    \end{align*}
Thus, the matrix of $T$ relative to the bases $B$ and $B'$ is
\[A=
\begin{bmatrix}
    0&1&-1&-1\\
    1&\frac{3}{4}&\frac{3}{2}&\frac{3}{2}
\end{bmatrix}
\]
\end{itemize}
\subsubsection*{Problem 44}
We construct vectors of the coffecients of $B$ and $B$ and find the matrix of $T$ relative to the bases $B$ and $B'$: 
\begin{align*}
    T\left(
        \begin{bmatrix}
            1\\
            0\\
            0
        \end{bmatrix}
    \right) = \begin{bmatrix}
        0\\
        0\\
        1\\
        0\\
        0\\
    \end{bmatrix} = (1) \begin{bmatrix}
        0\\
        0\\
        1\\
        0\\
        0\\
    \end{bmatrix},\text{ so} T[(1,0,0)]_B= \begin{bmatrix}
        0\\
        0\\
        1\\
        0\\
        0\\
    \end{bmatrix}.\\
    T\left(
        \begin{bmatrix}
            0\\
            1\\
            0
        \end{bmatrix}
    \right) = \begin{bmatrix}
        0\\
        0\\
        0\\
        1\\
        0\\
    \end{bmatrix} = (1) \begin{bmatrix}
        0\\
        0\\
        0\\
        1\\
        0\\
    \end{bmatrix},\text{ so} T[(1,0,0)]_B= \begin{bmatrix}
        0\\
        0\\
        0\\
        1\\
        0\\
    \end{bmatrix}.\\
    T\left(
        \begin{bmatrix}
            0\\
            0\\
            1
        \end{bmatrix}
    \right) = \begin{bmatrix}
        0\\
        0\\
        0\\
        0\\
        1\\
    \end{bmatrix} = (1) \begin{bmatrix}
        0\\
        0\\
        0\\
        0\\
        1\\
    \end{bmatrix},\text{ so} T[(1,0,0)]_B= \begin{bmatrix}
        0\\
        0\\
        0\\
        0\\
        1\\
    \end{bmatrix}.
\end{align*}
Thus, the matrix of $T$ relative to the bases $B$ and $B'$ is 
\[A=\begin{bmatrix}
    0&0&0\\
    0&0&0\\
    1&0&0\\
    0&1&0\\
    0&0&1
\end{bmatrix}.\]
\end{document}