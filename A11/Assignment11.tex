\documentclass[a4paper,12pt]{article}
\usepackage{graphicx, setspace, array, xcolor, mathtools, contour, tikz, amsthm, setspace, amsmath,amsfonts,amssymb, subcaption, fancyhdr, lipsum,multicol, geometry}
\usepackage{colortbl}
\geometry{a4paper, margin=1in}
\contourlength{0.5pt}
\usetikzlibrary{patterns}
\usepackage[english]{babel}
\renewcommand{\qedsymbol}{$\blacksquare$}
\definecolor{darkgreen}{rgb}{0.0, 0.5, 0.0}
\pagestyle{fancy}
\fancyhf{}
\renewcommand{\headrulewidth}{0.4pt}
\fancyhead[L]{\rightmark}
\fancyhead[R]{\thepage}
\title{Linear Algebra}
\author{Assignment 11\\ \\ Yousef A. Abood\\ \\ ID: 900248250}
\date{Jully 2025}
\setlength{\parindent}{0pt}
\singlespacing
\parskip=1mm
\setcounter{section}{6}
\setcounter{subsection}{1}
\onehalfspacing
\begin{document}
\maketitle
\noindent\makebox[\linewidth]{\rule{15cm}{0.4pt}}
\subsection{The Kernel and Range of a Linear Transformation}
\subsubsection*{Problem 7}
By definition, the zero vector of codomain is the polynomial\[0+0x\]. Comparing the zero vector of the codomain with the linear transformation we get \[0+0x=a_1+2a_2x\]. Thus, \[a_1=0, a2=0\]. But we see that $a_0$ is not in the image so it can be a free variable and we let it to be a real number $t$.
Therefore, \[ker(T)={t+0x+0x^2| t \in \mathbb{R}}\]
\subsubsection*{Problem 22}
\begin{itemize}
    \item [a)] To find the kernel we need to solve the system 
\[
\begin{bmatrix}
    4&1\\
    0&0\\
    2&-3
\end{bmatrix} \begin{bmatrix}
    x\\
    y
\end{bmatrix}=\begin{bmatrix}
    0\\
    0
\end{bmatrix}
\]
To solve this homogenous system, we start applying the EROs
\begin{align*}
    \begin{bmatrix}
    4&1\\
    0&0\\
    2&-3
\end{bmatrix}\xrightarrow{{R_2}\leftrightarrow{R_3}}
\begin{bmatrix}
    4&1\\
    2&-3\\
    0&0
\end{bmatrix} \xrightarrow{{\frac{1}{4}R_1}\to{R_1}}
\begin{bmatrix}
    1&\frac{1}{4}\\
    2&-3\\
    0&0
\end{bmatrix} \xrightarrow{{(R_2-2R_1)}\to{R_2}}
\begin{bmatrix}
    1&\frac{1}{4}\\
    0&\frac{-7}{2}\\
    0&0
\end{bmatrix} \xrightarrow{{\frac{-2}{7}R_2}\to{R_2}}
\begin{bmatrix}
    1&\frac{1}{4}\\
    0&1\\
    0&0
\end{bmatrix}
\end{align*}
We have that $y=0, x=0.$ So the kernel is the trivial space 
\[ker(T)=\left\{\begin{bmatrix}
0\\
0
\end{bmatrix}\right\}\]
    \item [b)] Using the fact that $nullity(T)=dim(ker(T))$. Then $nullity(T)=0,$ since the kernel only contains the trivial space. 
    \item [c)] $Range(T)=C(A)$, the span of the columns in $A$ that correspond to the pivot columns, so \\$Range(T)= span\left\{\begin{bmatrix}
    4\\
    0\\
    2
    \end{bmatrix}, \begin{bmatrix}
    1\\
    0\\
    -3
    \end{bmatrix}\right\}.$
    \item [d)] $Rank(T)=dim(Range(T))=dim(C(A))=2.$
\end{itemize}
\subsubsection*{Problem 40}
Observe that $T(x,y,z)=(x,y,0)$ can be written in the form $A\vec{x}=\vec{b}:$
\begin{align*}
    \begin{bmatrix}
      1 & 0 & 0 \\
      0 & 1 & 0 \\
      0 & 0 & 0
    \end{bmatrix} \begin{bmatrix}
        x\\
        y\\
        z
    \end{bmatrix} = \begin{bmatrix}
        x\\
        y\\
        0
    \end{bmatrix}.
\end{align*}
$A$ is clearly in REF, and has two pivot columns which are the $1^{st}, 2^{nd}$ columns, so $rank(A)=2.$ Using the fact that $rank(T)+nullity(T)=rank(A)+nullity(A)=n$ we deduce that $nullity(T)=3-2=1.$
\\
To find the kernel, we solve the system $A\vec{x}=\vec{b}$ when $\vec{b}=\vec{\textbf{0}_{3\times 3}}$. We see that $A$ is already in REF, so we don't need to apply the Gaussian elimination method. We get the system : $x=0, y=0$ and $z$ is a free varible since it corresponds to the third column which is a non-pivot column. So \begin{align*}
    ker(T)=N(A)= 
     \left\{
        \begin{bmatrix}
            0\\
            0\\
            t
        \end{bmatrix}: t \in \mathbb{R}
    \right\}
\end{align*}
Clearly, the geometric representation of the kernel is the set of all points in the $z-axis$ in $\mathbb{R}^3.$

To find the range, we use that $range(T)=C(A).$ Since $A$ is in REF, we pick the columns in $A$ that are pivot columns to get the basis of the column space. 
\begin{align*}
    C(A)= span\left\{
        \begin{bmatrix}
            1\\
            0\\
            0
        \end{bmatrix},\begin{bmatrix}
            0\\
            1\\
            0
        \end{bmatrix}
    \right\} =range(T)
\end{align*}
Which is interpretted geometrically as the set of every point in $xy-plane$ in $\mathbb{R}^3.$
\subsubsection*{Problem 45}
We use the theorem that states
\[rank(T)+nullity(T)=dim(domain(T)).\]
We already have that $rank(T)=4,$ and $dim(domain(T))=2 \times 4=8.$ Hence, \[nullity(T)=dim(domain(T))-rank(T)=8-4=4.\]
\subsubsection*{Problem 57} By definition of the linear transformation, we have \[T(a_0+a_1x+a_2x^2+a_3x^3+a_4x^4)=a_1+2a_2x+3a_3x^2+4a_4x^3\]
The zero vector of the codomain is $0+0x+0x^2+0x^3$. We compare the zero vector to the image of the linear transformation and get that in the kernal $a_1=0, a_2=0, a_3=0, a_4=0.$
The ramaining is $a_0$ which we take as a free variable as its value does not affect the linear transformation. Thus, 
\[ker(T)=\{t+0x+0x^2+0x^3+0x^4| t \in \mathbb{R}\}.\]
\subsubsection*{Problem 58}
We solve the following integral
\begin{align*}
    \int_{0}^{1}{a_0+a_1x+a_2x^2}&= \left[a_0x+\frac{a_1}{2}x^2+\frac{a_2}{3}x^3 \right]_{0}^{1}\\
                                 &=a_0+\frac{a_1}{2}+\frac{a_2}{3}.
\end{align*}
By definition of the kernel, the zero vector in $\mathbb{R}$ is $\textbf{0}.$ Then, we need to solve \[a_0+\frac{a_1}{2}+\frac{a_2}{3}=0.\]
Take $a_1, a_2$ as free variables $s,t$ respectivily. Hence, \[a_0=\frac{-1}{2}s-\frac{1}{3}t.\] and 
\begin{align*}
p(x)&=\frac{-1}{2}s-\frac{1}{3}t+sx+tx^2\\
    &=s(\frac{-1}{2}+x)+t(-\frac{1}{3}+x^2)
\end{align*}
We see that we get a set of linear combinations of polynomials which form the kernel. Therefore,
\[ker(T)=span((\frac{-1}{2}+x),(-\frac{1}{3}+x^2))\]
\end{document}