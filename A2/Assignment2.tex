\documentclass[a4paper,12pt]{article}
\usepackage{graphicx, setspace, array, xcolor
, mathtools, contour, tikz, amsthm, setspace, amsmath,amsfonts,amssymb, subcaption, fancyhdr, lipsum,multicol}
\usepackage[a4paper, left=2.5cm, right=2.5cm, top=3cm, bottom=2cm]{geometry}
\usepackage{colortbl} % Add this to your preamble
\contourlength{0.5pt}
\usetikzlibrary{patterns}
\usepackage[english]{babel}
\renewcommand{\qedsymbol}{$\blacksquare$}
\definecolor{darkgreen}{rgb}{0.0, 0.5, 0.0} % Custom dark green
\pagestyle{fancy}
\fancyhf{}
\renewcommand{\headrulewidth}{0.4pt}
\fancyhead[L]{\rightmark}
\fancyhead[R]{\thepage}
\title{Linear Algebra}
\author{Assignment 2\\ \\ Yousef A. Abood\\ \\ ID: 900248250}
\date{June 2025}
\setlength{\parindent}{0pt}
\singlespacing
\parskip=1mm
\setcounter{section}{1}
\onehalfspacing
\begin{document}
\maketitle
\noindent\makebox[\linewidth]{\rule{15cm}{0.4pt}}
\subsection{Introduction to Systems of Linear Equations}
\subsubsection*{Problem 2}
\textit{Linear equations are written in the form : $a_{1}x_{1}+a_{2}x_{2}+a_{3}x_{3}+ \dots +a_{n}x_{n}$}\\
\textcolor{red}{Not a linear equation.} It violates the form as it has the term $(-4xy)$
\subsubsection*{Problem 3}
\textcolor{red}{Not a linear equation.} It violates the form as it has the terms $\frac{3}{y}, \frac{2}{x}.$
\subsubsection*{Problem 6}
\textcolor{red}{Not a linear equation.} It violates the form as it has the terms $(\cos 3) x.$
\subsubsection*{Problem 7}
Let $x=t \iff y= \frac{1}{2}t.$ The solution set is $\{(t,\frac{1}{2}t)|t \in \mathbb{R}\}$.
\subsubsection*{Problem 9}
Let $x=t, y=s \iff z=1-t-s.$ The solution set is $\{(t,s,1-s-t)|t,s \in \mathbb{R}\}$
\subsubsection*{Problem 52}
\[
\begin{array}{rrrrrrr}
x_1 &      & +\,4x_3 & = & 13 \\
4x_1 & -\,2x_2 & +\,x_3 & = & 7 \\
2x_1 & -\,2x_2 & -\,7x_3 & = & -19\\
\end{array}\] {Add the second equation to -4 of the first equation producing new second equation.}
\[
\begin{array}{rrrrrrr}
x_1 &      & +\,4x_3 & = & 13 \\
 & -\,2x_2 & -\,15x_3 & = & -45 \\
2x_1 & -\,2x_2 & -\,7x_3 & = & -19\\
\end{array}\]{Add the third equation to -2 of the first equation producing new third equation.}
\[\begin{array}{rrrrrrr}
x_1 &      & +\,4x_3 & = & 13 \\
 & -\,2x_2 & -\,15x_3 & = & -45 \\
& -\,2x_2 & -\,15x_3 & = & -45\\
\end{array}\]
We notice the second and the third equations are identical, so we have system of just two equations. We can set the common variable $x_3$ to $t$, so we have : \\ $x_1=13-4t, x_2=\frac{45}{2}-\frac{15}{2}t$, and the solution set is $\{(13-4t,\frac{45}{2}-\frac{15}{2}t,t)| t \in \mathbb{R}\}$
\subsubsection*{Problem 85}
\[
\begin{array}{rrrrrrr}
x &+& y&+& kz & = & 3 \\
x &+& ky&+& z & = & 2 \\
kx&+& y &+& z & = & 1\\
\end{array}\] {Add the second equation to -1 of the first equation producing new second equation.}
\[
\begin{array}{rrrrrrr}
x &+& y&+& kz & = & 3 \\
 && (k-1)y&+& (1-k)z & = & -1 \\
kx&+& y &+& z & = & 1\\
\end{array}\]{Add the third equation to -k of the first equation producing new third equation.}
\[\begin{array}{rrrrrrr}
x &+& y&+& kz & = & 3 \\
 && (k-1)y&+& (1-k)z & = & -1 \\
&& (1-k)y &+& (1-k^2)z & = & 1-3k\\
\end{array}\] {Add the third equation to the second equation producing new third equation.}
\[\begin{array}{rrrrrrr}
x &+& y&+& kz & = & 3 \\
 && (k-1)y&+& (1-k)z & = & -1 \\
&&&& (-k^2-k+2)z & = &-3k\\
\end{array}\]
So $z=\frac{-3k}{(1-k)(2+k)}$. We substitute in the second equation getting $(k-1)y+\frac{-3k}{2+k}=-1$ and $y= \frac{-1}{k-1}+ \frac{3k}{(2+k)(k-1)} = \frac{-1}{k-1}+z$. Now, we can say that $x = 3+\frac{1}{k-1}-z-kz=3+\frac{1}{k-1}-\frac{-3k}{(1-k)(2+k)}-\frac{-3k^2}{(1-k)(2+k)} = 3+\frac{1}{k-1}(1-\frac{3k}{2+k}(1+k)).$ Hence, all real values for $k$ will satisfy $x$ except $k=1, k= -2$. Therefore, to get the system of linear of equations which does not have a unique solution then $k=-2, or k=1.$
\subsection{Gaussian Elimination and Gauss-Jordan Elimination}
\subsubsection*{Problem 21}
\textcolor{red}{It is not in the row-echelon form.} The first non-zero entry in the first row is $-2$.
\subsubsection*{Problem 23}
\textcolor{red}{It is not in the row-echelon form.} The first non-zero entry in the third row is $2$.
\subsubsection*{Problem 29}
We keep applying the elementry row operations untill we reach the row-echelon form.
\begin{align*}\begin{bmatrix}
  -3 & 5 & -22 \\
  3 & 4 & 4 \\
  4 & -8 & 32
\end{bmatrix}
&\xrightarrow{\frac{-1}{3}R_1\to R_1}& &\begin{bmatrix}
  1 & \frac{-5}{3} & \frac{22}{3} \\
  3 & 4 & 4 \\
  4 & -8 & 32
\end{bmatrix} &\xrightarrow{(R_2-3R_1)\to R_2}& &\begin{bmatrix}
  1 & \frac{-5}{3} & \frac{22}{3} \\
  0 & 9 & 18 \\
  4 & -8 & 32
\end{bmatrix}\\ &\xrightarrow{(R_3-4R_1)\to R_2}& &\begin{bmatrix}
  1 & \frac{-5}{3} & \frac{22}{3} \\
  0 & 9 & 18 \\
  0 & \frac{-4}{3} & \frac{8}{3}
\end{bmatrix} &\xrightarrow{\frac{1}{9}R_2\to R_2}& &\begin{bmatrix}
  1 & \frac{-5}{3} & \frac{22}{3} \\
  0 & 1 & 2 \\
  0 & \frac{-4}{3} & \frac{8}{3}
\end{bmatrix} \\ &\xrightarrow{(\frac{4}{3}R_2+R_3)\to R_3}& &\begin{bmatrix}
  1 & \frac{-5}{3} & \frac{22}{3} \\
  0 & 1 & 2 \\
  0 & 0 & \frac{16}{3}
\end{bmatrix} &\xrightarrow{\frac{3}{16}R_3\to R_3}& &\begin{bmatrix}
  1 & \frac{-5}{3} & \frac{22}{3} \\
  0 & 1 & 2 \\
  0 & 0 & 1
\end{bmatrix}
\end{align*}
From the third row, we get the equation $0x+0y=1$, which has no solutions. Hence, \textcolor{red}{the system has no solutions and it is inconsistent}
\subsubsection*{Problem 37}
We keep applying the elementry row operations untill we reach the row-echelon form.
\begin{align*}
\begin{bmatrix}
  3 & 3 & 12 &6\\
  1 & 1 & 4 &2\\
  2 & 5 & 20 &10\\
  -1& 2 & 8 & 4
\end{bmatrix}  &\xrightarrow{\frac{1}{3}R_1 \to R_1}& \begin{bmatrix}
  1 & 1 & 4 &2\\
  1 & 1 & 4 &2\\
  2 & 5 & 20 &10\\
  -1& 2 & 8& 4&
\end{bmatrix} &\xrightarrow{(R_2-R_1)\to R_2}& \begin{bmatrix}
  1 & 1 & 4 &2\\
  0 &  0& 0 &0\\
  2 & 5 & 20 &10\\
  -1& 2 & 8& 4&
\end{bmatrix}\\ &\xrightarrow{R_2 \leftrightarrow R_4}& \begin{bmatrix}
  1 & 1 & 4 &2\\
  -1 &  2& 8 &4\\
  2 & 5 & 20 &10\\
  0& 0 & 0& 0
\end{bmatrix} &\xrightarrow{(R_1+R_2)\to R_2}& \begin{bmatrix}
  1 & 1 & 4 &2\\
  0 &  3& 12 &6\\
  2 & 5 & 20 &10\\
  0& 0 & 0& 0
\end{bmatrix}\\ &\xrightarrow{(R_3-2R_1)\to R_3}& \begin{bmatrix}
  1 & 1 & 4 &2\\
  0 &  3& 12 &6\\
  0 & 3 & 12 &6\\
  0& 0 & 0& 0
\end{bmatrix} &\xrightarrow{(R_3-R_2)\to R_3}& \begin{bmatrix}
  1 & 1 & 4 &2\\
  0 &  3& 12 &6\\
  0 & 0& 0 &0\\
  0& 0 & 0& 0
\end{bmatrix} \\ &\xrightarrow{(R_2-3R_1)\to R_2}& \begin{bmatrix}
  1 & 1 & 4 &2\\
  0 &  0& 0 &0\\
  0 & 0& 0 &0\\
  0& 0 & 0& 0
\end{bmatrix}
\end{align*}
From the last operation, the equation $x+y+4z=2$ is row equivalent to the system of linear equations. We can conclude that we have infinitely many solutions. To get the solution set, we set $x=s, y=t \iff z= \frac{1}{4}(2-s-t)$. Therefore, the solution set for this system is $\{(s,t,\frac{1}{4}(2-s-t))| s,t \in \mathbb{R}\}$
\subsubsection*{Problem 44}
We can see that this matrix is in the reduced row-echelon form. So the equations of this system are $x=0, y+z=0$, and it is clear that $y=-z$. To get the solution set, we know that the fourth variable $w$ is a free variable, so we let $w=s, y=t$. Therefore, the soultion set is $\{(0,t,-t,s)| s,t\in \mathbb{R}\}.$
\subsubsection*{Problem 50}
\begin{itemize}
  \item [a)] Three equations and 2 variables.
  \item [b)] We keep applying the elementry row operations: \begin{align*}
    \begin{bmatrix}
      2 & -1 & 3 \\
      -4 & 2 & k \\
      4 & -2 & 6
    \end{bmatrix} &\xrightarrow{\frac{1}{2}R_1 \to R_1} \begin{bmatrix}
      1 & \frac{-1}{2} & \frac{3}{2}\\
      -4 & 2 & k \\
      4 & -2 & 6
    \end{bmatrix} \xrightarrow{(4R_1+R_2)\to R_2} \begin{bmatrix}
      1 & \frac{-1}{2} & \frac{3}{2}\\
      0 & 0 & 6+k \\
      4 & -2 & 6
    \end{bmatrix} \\ &\xrightarrow{(-4R_1+R_3)\to R_3} \begin{bmatrix}
      1 & \frac{-1}{2} & \frac{3}{2}\\
      0 & 0 & 6+k \\
      0 & 0 & 0
    \end{bmatrix} 
  \end{align*}
From the previous steps, to make the system consistent, $6+k$ is a free variable. Let $6+k =$ any real number $i$, then $k$ can be any real number $i-6$.
\end{itemize}
\textit{In a homogeneous system}
\begin{itemize}
  \item [a)] We have three equations in three variable.
  \item [b)] We apply the same previously applied EROs getting $\begin{bmatrix}
      1 & \frac{-1}{2} & \frac{3}{2}\\
      0 & 0 & 6+k \\
      0 & 0 & 0
    \end{bmatrix}.$ We know the augmented matrix is $\begin{bmatrix}
      1 & \frac{-1}{2} & \frac{3}{2}&0\\
      0 & 0 & 6+k &0\\
      0 & 0 & 0 &0
    \end{bmatrix}.$ so $(6+k)z=0$ and $k=-6$.
\end{itemize}
\subsubsection*{Problem 51}
\begin{align*}\begin{bmatrix}
    1 & 1 & 0 & 2 \\
    0 & 1 & 1 & 2\\
    1 & 0 & 1 & 2
  \end{bmatrix} \xrightarrow{(R_3-R_1)\to R_3} \begin{bmatrix}
    1 & 1 & 0 & 2 \\
    0 & 1 & 1 & 2\\
    0 & -1 & 1 & 0
  \end{bmatrix} \xrightarrow{(R_3+R_2)\to R_3} \begin{bmatrix}
    1 & 1 & 0 & 2 \\
    0 & 1 & 1 & 2\\
    0 & 0 & 2 & 2
  \end{bmatrix}
\end{align*}
From this we see $z=1, y=1, x=1$. But this is the solution of the system of the first three equations and we know that they all intersect in exactly one point $(1,1,1).$
\begin{itemize}
    \item [a)] For the system of these four equations to have one solution. The solution of the first three equations must satisfy the fourth equation. So we have $a+b+c=0$, and we know that $a,b,c$ are real numbers. So we consider $b,c$ free variables and $a=-b-c$.
  \item [b)] Since $a,b,c$ are real numbers, any values of $a,b,c$ that do not satisfy $a=-b-c$ is considered values where the system has no solutions.
  \item [c)] This is impossible. The system of the first three equations has only one solutions. Therefore, no other solutions can be found for this system and any other bigger system that includes the same equations. 
\end{itemize}
\subsubsection*{Problem 61}
Suppose we have two equations of three variables. It is clear that each expresses a plane in a 3d-space. We know that if the two planes are parallel then they do not intersect and the system has no solution. Example : $x+z+y=1, x+z+y=2$, applying the EROs for the augmented matrix of this system we have $0x+0y=1$, which has no solution.
\subsubsection*{Problem 62}
No, let $A=\begin{bmatrix}
  1 & 2 & 3 \\
  4 & 5 & 6 \\
  7 & 8 & 9
\end{bmatrix}.$\\
We can proceed by keep applying EROs :
\begin{align*}
  \begin{bmatrix}
    1 & 2 & 3 \\
    4 & 5 & 6 \\
    7 & 8 & 9
  \end{bmatrix} &\xrightarrow{{(R_2-4R_1)}	\to{R_2}}
  \begin{bmatrix}
    1 & 2 & 3 \\
    0 & -3 & -6 \\
    7 & 8 & 9
  \end{bmatrix} \xrightarrow{{(R_3-7R_1)}	\to{R_3}}
   \begin{bmatrix}
    1 & 2 & 3 \\
    0 & -3 & -6 \\
    0 & -6 & -12
  \end{bmatrix}\\ &\xrightarrow{{\frac{-1}{3}R_2}	\to{R_2}}  \begin{bmatrix}
    1 & 2 & 3 \\
    0 & 1 & 2 \\
    0 & -6 & -12
  \end{bmatrix} \xrightarrow{{(R_3+6R_1)}	\to{R_3}}
  \begin{bmatrix}
    1 & 2 & 3 \\
    0 & 1 & 2 \\
    0 & 0 & 0
  \end{bmatrix}.
\end{align*}\\
But we can change the result by starting with interchanging any two rows :
\begin{align*}
  \begin{bmatrix}
    1 & 2 & 3 \\
    4 & 5 & 6 \\
    7 & 8 & 9
  \end{bmatrix} &\xrightarrow{{R_2}\leftrightarrow {R_1}}
  \begin{bmatrix}
    4 & 5 & 6 \\
    1 & 2 & 3 \\
    7 & 8 & 9
  \end{bmatrix} \xrightarrow{{\frac{1}{4}R_1}	\to{R_1}} 
  \begin{bmatrix}
    1 & \frac{5}{4} & \frac{3}{2} \\
    1 & 2 & 3 \\
    7 & 8 & 9
  \end{bmatrix} \xrightarrow{{(R_2-R_1)}	\to{R_2}}
  \begin{bmatrix}
    1 & \frac{5}{4} & \frac{3}{2} \\
    0 & \frac{3}{4} & \frac{3}{2} \\
    7 & 8 & 9
  \end{bmatrix}\\ &\xrightarrow{{\frac{4}{3}R_2}	\to{R_2}}
  \begin{bmatrix}
    1 & \frac{5}{4} & \frac{3}{2} \\
    0 & 1 & 2 \\
    7 & 8 & 9
  \end{bmatrix} \xrightarrow{{(R_3-7R_1)}	\to{R_3}} 
  \begin{bmatrix}
    1 & \frac{5}{4} & \frac{3}{2} \\
    0 & 1 & 2 \\
    0 & \frac{-3}{4} & \frac{-3}{2}
  \end{bmatrix}\\ &\xrightarrow{{(\frac{-4}{3}R_3)}	\to{R_3}}
  \begin{bmatrix}
    1 & \frac{5}{4} & \frac{3}{2} \\
    0 & 1 & 2 \\
    0 & 1 & 2
  \end{bmatrix} \xrightarrow{{(R_3-R_2)}	\to{R_3}}
   \begin{bmatrix}
    1 & \frac{5}{4} & \frac{3}{2} \\
    0 & 1 & 2 \\
    0 & 0 & 0.
  \end{bmatrix}
\end{align*}

As we see above, we obtained two different matrices in the row-echelon form from one matrix using the elementary row operations. Hence, the matrix does not have a unique row-echelon form.
\subsubsection*{Problem 66}
From the given system, $x=\lambda y$, so $\lambda(2 \lambda + 9)y-5y=0=y(2 {\lambda}^2+9 \lambda -5)$. We want the non-trivial solution, so $y \neq 0$. Hence, $2 {\lambda}^2+9 \lambda -5=0$, then $\lambda =\frac{1}{2} or \lambda = -5.$ Therefore, $\lambda =\frac{1}{2} or \lambda = -5.$ are the values for which the system has no trivial solutions.
\end{document}