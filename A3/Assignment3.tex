\documentclass[a4paper,12pt]{article}
\usepackage{graphicx, setspace, array, xcolor
, mathtools, contour, tikz, amsthm, setspace, amsmath,amsfonts,amssymb, subcaption, fancyhdr, lipsum,multicol}
\usepackage[a4paper, left=2.5cm, right=2.5cm, top=3cm, bottom=2cm]{geometry}
\usepackage{colortbl}
\contourlength{0.5pt}
\usetikzlibrary{patterns}
\usepackage[english]{babel}
\renewcommand{\qedsymbol}{$\blacksquare$}
\definecolor{darkgreen}{rgb}{0.0, 0.5, 0.0}
\pagestyle{fancy}
\fancyhf{}
\renewcommand{\headrulewidth}{0.4pt}
\fancyhead[L]{\rightmark}
\fancyhead[R]{\thepage}
\title{Linear Algebra}
\author{Assignment 3\\ \\ Yousef A. Abood\\ \\ ID: 900248250}
\date{June 2025}
\setlength{\parindent}{0pt}
\singlespacing
\parskip=1mm
\setcounter{section}{2}
\setcounter{subsection}{2}
\onehalfspacing
\begin{document}
\maketitle
\noindent\makebox[\linewidth]{\rule{15cm}{0.4pt}}
\subsection{The inverse of a matrix}
\subsubsection*{Problem 5} 
We need to show that $AB=I_3$, we can see that clearly as follows:
\begin{align*}
    \begin{bmatrix}
      -2 & 2 & 3 \\
      1 & -1 & 0 \\
      0 & 1 & 4
    \end{bmatrix} \begin{bmatrix}
      \frac{-4}{3} & \frac{-5}{3} & 1 \\
      \frac{-4}{3} & \frac{-8}{3} & 1 \\
      \frac{1}{3} & \frac{2}{3} & 0
    \end{bmatrix} = \begin{bmatrix}
      1 & 0 & 0 \\
      0 & 1 & 0 \\
      0 & 0 & 1
    \end{bmatrix}
\end{align*}
\subsubsection*{Problem 12}
We keep applying the same EROs to booth the matrix and the identity matrix of the same size to arrive to the inverse as follows: 
\begin{align*}
    \left [\begin{array}{cc|cc}
     -1 & 1 & 1 & 0 \\
      3 & -3& 0 & 1   
    \end{array} \right ] \xrightarrow{{-R_1}	\to{R_1}}
    \left [\begin{array}{cc|cc}
     1 & -1 & -1 & 0 \\
      3 & -3& 0 & 1   
    \end{array} \right ] \xrightarrow{{(R_2-3R_1)}\to{R_2}}
    \left [\begin{array}{cc|cc}
     1 & -1 & -1 & 0 \\
      0 & 0& 3 & 1   
    \end{array} \right ].
\end{align*}
We can see we got a matrix with row of zeros. Therefore, this inverse is not invertible.
\subsubsection*{Problem 23}
We keep applying the same EROs to booth the matrix and the identity matrix of the same size to arrive to the inverse as follows: 
\begin{align*}
    \left [
\begin{array}{ccc|ccc}
  1 & 0 & 0 & 1 & 0 & 0\\
  3 & 4 & 0 & 0 & 1 & 0\\
  2 & 5 & 5 & 0 & 0 & 1
\end{array} \right ] \xrightarrow{{(R_2-3R_1)}\to{R_2}}
&\left [
\begin{array}{ccc|ccc}
  1 & 0 & 0 & 1 & 0 & 0\\
  0 & 4 & 0 & -3 & 1 & 0\\
  2 & 5 & 5 & 0 & 0 & 1
\end{array} \right ]\\ \xrightarrow{{(R_3-2R_1)}\to{R_3}}
&\left [
\begin{array}{ccc|ccc}
  1 & 0 & 0 & 1 & 0 & 0\\
  0& 4 & 0 & -3 & 1 & 0\\
  0 & 5 & 5 & -2 & 0 & 1
\end{array} \right ] \\ \xrightarrow{{\frac{1}{4}R_2}\to{R_2}}
&\left [
\begin{array}{ccc|ccc}
  1 & 0 & 0 & 1 & 0 & 0\\
  0& 1 & 0 & \frac{-3}{4} & \frac{1}{4} & 0\\
  0 & 5 & 5 & -2 & 0 & 1
\end{array} \right ] \\ \xrightarrow{{(R_3-5R_2)}	\to{R_3}}
&\left [
\begin{array}{ccc|ccc}
  1 & 0 & 0 & 1 & 0 & 0\\
  0& 1 & 0 & \frac{-3}{4} & \frac{1}{4} & 0\\
  0 & 0 & 5 & \frac{7}{4} & \frac{-5}{4} & 1
\end{array} \right ]\\ \xrightarrow{{\frac{1}{5}R_3}\to{R_3}}
&\left [
\begin{array}{ccc|ccc}
  1 & 0 & 0 & 1 & 0 & 0\\
  0& 1 & 0 & \frac{-3}{4} & \frac{1}{4} & 0\\
  0 & 0 & 1 & \frac{7}{20} & \frac{-1}{4} & \frac{1}{5}
\end{array} \right ].
\end{align*}
So the inverse of the matrix = $\begin{bmatrix}
    1 & 0 & 0 \\
    \frac{-3}{4} & \frac{1}{4} & 0 \\
    \frac{7}{20} & \frac{-1}{4} & \frac{1}{5}
  \end{bmatrix}.$
\subsubsection*{Problem 47}
\begin{itemize}
    \item [b)] Suppose the matrix of the coefficients $A$, the matrix of varibles $X$, and the matrix of constants $B$, so $AX=B$. To solve the equation, we multiply each side by the $A^{-1}$ from the left. We get $A^{-1}$ as follows:
    \begin{align*}
        \left [\begin{array}{ccc|ccc}
            1&2&1&1&0&0\\
            1&2&-1&0&1&0\\
            1&-2&1&0&0&1
        \end{array} \right ] \xrightarrow{{(R_2-R_1)}\to{R_2}}
        &\left [\begin{array}{ccc|ccc}
            1&2&1&1&0&0\\
            0&0&-2&-1&1&0\\
            1&-2&1&0&0&1
        \end{array} \right ]\\ \xrightarrow{{(R_3-R_1)}\to{R_3}}
        &\left [\begin{array}{ccc|ccc}
            1&2&1&1&0&0\\
            0&0&-2&-1&1&0\\
            0&-4&0&-1&0&1
        \end{array} \right ]\\ \xrightarrow{{R_2}\leftrightarrow{R_3}}
         &\left [\begin{array}{ccc|ccc}
            1&2&1&1&0&0\\
            0&-4&0&-1&0&1\\
            0&0&-2&-1&1&0
        \end{array} \right ]\\ \xrightarrow{{\frac{-1}{4}R_2}\to{R_2}}
        &\left [\begin{array}{ccc|ccc}
            1&2&1&1&0&0\\
            0&1&0&\frac{1}{4}&0&\frac{-1}{4}\\
            0&0&-2&-1&1&0
        \end{array} \right ]\\ \xrightarrow{{(R_1-2R_2)}\to{R_1}}
        &\left [\begin{array}{ccc|ccc}
            1&0&1&\frac{1}{2}&0&\frac{1}{2}\\
            0&1&0&\frac{1}{4}&0&\frac{-1}{4}\\
            0&0&-2&-1&1&0
        \end{array} \right ]\\ \xrightarrow{{\frac{-1}{2}R_3}\to{R_3}}
        &\left [\begin{array}{ccc|ccc}
            1&0&1&\frac{1}{2}&0&\frac{1}{2}\\
            0&1&0&\frac{1}{4}&0&\frac{-1}{4}\\
            0&0&1&\frac{1}{2}&\frac{-1}{2}&0
        \end{array} \right ]\\ \xrightarrow{{(R_1-R_3)}\to{R_1}}
        &\left [\begin{array}{ccc|ccc}
            1&0&0&0&\frac{1}{2}&\frac{1}{2}\\
            0&1&0&\frac{1}{4}&0&\frac{-1}{4}\\
            0&0&1&\frac{1}{2}&\frac{-1}{2}&0
        \end{array} \right ]
    \end{align*}
\end{itemize}
\subsubsection*{Problem 55}
$A$ is singular iff its determinant is equal to zero. Hence, $det(A)= -12+2x = 0$, when it is singular, so $x=6$. 
\subsubsection*{Problem 68}
\begin{proof}
    Suppose $A,B,C$ are square matrices and $ABC=I$. By properties of matrix multiplication, we use the associativity to get $(AB)C=I$, and we know that the result of the matrix multiplication is a matrix. Since we multiply square matrices we will not care about size as it will not change through multiplication by other square matrices of the same size.
    Hence, $(AB)C=DC=I$. By definition of the inverse matrix, we know that $C$ is clearly invertible and $D=C^{-1}.$ By a theorem proved in class, if $(AB)$ is invertible, then $A$ is invertible and $B$ is invertible. For the second part, start with $ABC=I$, we know each of these matrices is invertible, so we multiply from the left by $A^{-1}$ and from right by $C^{-1}$ to get $B=A^{-1}IC^{-1}=A^{-1}C^{-1}.$ Hence, $B^{-1}=(A^{-1}C^{-1})^{-1}=CA.$
\end{proof}
\subsubsection*{Problem 76}
$\begin{bmatrix}
      1 & 2 \\
      -2 & 1
    \end{bmatrix}.$
\begin{itemize}
    \item [a)] $A^2=\begin{bmatrix}
      -3 & 4 \\
      -4 & -3
    \end{bmatrix}$, $2A \begin{bmatrix}
      2 & 4 \\
      -4 & 2
    \end{bmatrix}.$\\
    So $A^2-2A+5I= \begin{bmatrix}
      0 & 0 \\
      0 & 0
    \end{bmatrix} = \textbf{0}.$
    \item [b)] $\frac{1}{5}(2I-A)=\frac{1}{5}(\begin{bmatrix}
      2 & 0 \\
      0 & 2
    \end{bmatrix}-\begin{bmatrix}
      1 & 2 \\
      -2 & 1
    \end{bmatrix}) = \begin{bmatrix}
      \frac{1}{5} & \frac{-2}{5} \\
      \frac{2}{5} & \frac{1}{5}
    \end{bmatrix}$. \\
   \\ We can get $A^{-1}$ quickly by interchanging the main daigonal entries and multiply \\ \\ entries of other diagonal by $-1$ and multiply the new matrix by the determinant of the old one. So we get $\frac{1}{1-(-4)}\begin{bmatrix}
      1 & -2 \\
      2 & 1
    \end{bmatrix} = \frac{1}{5}(2I-A).$ \\

    \textbf{Due to time constraints, I'll complete using scanned pen and paper solution.}
    \item [c)]
\end{itemize}

\end{document}