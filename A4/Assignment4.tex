\documentclass[a4paper,12pt]{article}
\usepackage{graphicx, setspace, array, xcolor
, mathtools, contour, tikz, amsthm, setspace, amsmath,amsfonts,amssymb, subcaption, fancyhdr, lipsum,multicol}
\usepackage[a4paper, left=2.5cm, right=2.5cm, top=3cm, bottom=2cm]{geometry}
\usepackage{colortbl}
\contourlength{0.5pt}
\usetikzlibrary{patterns}
\usepackage[english]{babel}
\renewcommand{\qedsymbol}{$\blacksquare$}
\definecolor{darkgreen}{rgb}{0.0, 0.5, 0.0}
\pagestyle{fancy}
\fancyhf{}
\renewcommand{\headrulewidth}{0.4pt}
\fancyhead[L]{\rightmark}
\fancyhead[R]{\thepage}
\title{Linear Algebra}
\author{Assignment 4\\ \\ Yousef A. Abood\\ \\ ID: 900248250}
\date{June 2025}
\setlength{\parindent}{0pt}
\singlespacing
\parskip=1mm
\setcounter{section}{3}
\setcounter{subsection}{0}
\onehalfspacing
\begin{document}
\maketitle
\noindent\makebox[\linewidth]{\rule{15cm}{0.4pt}}
\subsection{The Determinant of a Matrix}
\subsubsection*{Problem 12}
$
\left |\begin{array}{cc}
    \lambda-2&0\\
    4   & \lambda-4
\end{array}\right | = (\lambda-2)(\lambda-4)-4 \times 0= \lambda^2 -6\lambda+8.
$
\subsubsection*{Problem 25}
$
x \left |\begin{array}{cc}
    2&0\\
    1&1
\end{array}\right | - y \left |\begin{array}{cc}
    3&0\\
    1&1
\end{array}\right |-\left |\begin{array}{cc}
    3&2\\
    1&1
\end{array}\right | = x(2-0)-y(3-0)-(3-2)=2x-3y-1.
$
\subsubsection*{Problem 51}
$
\lambda \left |\begin{array}{cc}
    \lambda+1&2\\
    1& \lambda
\end{array}\right | = \lambda (\lambda(\lambda +1)-2)=\lambda (\lambda^2 +\lambda-2)=\lambda ((\lambda+2)(\lambda-1))=0.
$ Since the product is zero, then $\lambda=0$ or $\lambda=1$ or $\lambda=0.$
\subsubsection*{Problem 68}

\begin{align*}
    \begin{vmatrix}
        1&1&1\\
        a&b&c\\
        a^3&b^3&c^3
    \end{vmatrix} &= \begin{vmatrix}
        1&1&1\\
        0&b-a&c-a\\
        0&b^3-a^3&c^3-a^3
    \end{vmatrix} \quad \quad \quad \quad \quad ((R_3-a^3R_1)\to R_3\text{ and } (R_2-aR_1)\to R_2)
    \\ &= (b-a)(c^3-a^3)-(c-a)(b^3-a^3)\\&=(b-a)(c-a)(c^2+a^2+ca)-(c-a)(b-a)(b^2+a^2+ba)
    \\ &= (c-a)(b-a)(c^2-b^2+a(c-b))
    \\ &= (c-a)(b-a)((c-b)(c+b)+a(c-b))
    \\ &= (c-a)(b-a)(c-b)(a+b+c)
    \\ &= (a-b)(b-c)(c-a)(a+b+c)
\end{align*}


\subsection{Determinants and Elementary Operations}
\subsubsection*{Problem 23}
\begin{align*} det\begin{pmatrix}
    5&1&0&1\\
    1&0&-1&-1\\
    2&0&1&2\\
    -1&0&3&1
\end{pmatrix} =
    -\begin{vmatrix}
        1&-1&-1\\
        2&1&2\\
        -1&3&1
    \end{vmatrix} &= -\begin{vmatrix}
        1&2\\
        3&1
    \end{vmatrix} - \begin{vmatrix}
        2&2\\
        -1&1
    \end{vmatrix} + \begin{vmatrix}
        2&1\\
        -1&3
    \end{vmatrix}\\ &= -1+6-2-2+6+1=8.
\end{align*}
\subsubsection*{Problem 44}
\begin{proof}
    We compute the determinant using cofactor expansion as follows:
    \begin{align*}
        det\begin{pmatrix}
            1+a & 1 & 1\\
            1 & 1+b & 1\\
            1 & 1 & 1+c
        \end{pmatrix} &= (1+a) \begin{vmatrix}
            1+b& 1\\
            1 & 1+c
        \end{vmatrix} - \begin{vmatrix}
            1&1\\
            1&1+c
        \end{vmatrix} + \begin{vmatrix}
            1&1\\
            1+b&1
        \end{vmatrix}\\
        &= (1+a)((1+b)(1+c)-1)-(1+c-1)+(1-1-b)\\
        &=(1+b+a+ab)(1+c)-(1+a)-c-b\\
        &=1+b+a+ab+c+bc+ac+abc-1-a-c-b\\
        &=ab+bc+ac+abc = \frac{abc}{abc}(ab+bc+ac+abc)\\
        &=abc(\frac{1}{c}+\frac{1}{a}+\frac{1}{b}+1) =abc(1+\frac{1}{a}+\frac{1}{b}+\frac{1}{c})
    \end{align*}
That satisfies the proof.
\end{proof}
\subsection{Properties of Determinants}
\subsubsection*{Problem 45}
\begin{itemize}
    \item [a)] We know $|A^T|=|A|$, so $|A^T|= 5\begin{vmatrix}
        -3&0\\
        -1&2
    \end{vmatrix} = 5(-6)=-30.$
    \item [b)] By multiplicativity of a determinant, $det(A^2)=det(AA)=det(A)det(A)=\\-30\times-30=900.$
    \item [c)] By multiplicativity of a determinant, $det(A^TA)=det(A^T)det(A)$. And we know $det(A^T)=det(A)$. So, $|A^TA|=-30\times-30=900.$
    \item [d)] $A$ is a matrix of size $3\times3$, so $|2A|=2^3|A|=8\times -30=-240.$
    \item [e)]  $|A^{-1}|=\frac{1}{|A|}=\frac{-1}{30}.$
\end{itemize}
\subsubsection*{Problem 59}
\begin{proof}
    By Multiplicativity of Determinant, $det(AB)=det(A)\times det(B)=det(I)=1.$ We clearly see that if either $det(A),det(B)$ is zero, then it is impossible for their product to be $1$. Therefore, we showed that $|A|\neq 0, |B| \neq 0.$
\end{proof}
\subsubsection*{Problem 68}
\begin{proof}
    
We know that if a matrix is invertible then its determinant cannot be zero. And we know, by multiplicativity of determinant, that $det(A^{10})=(det(A))^{10}.$ Suppose $A$ is invertible, then $|A|\neq 0$ and clearly $|A|^{10}\neq 0.$ A contradiction. Therefore, $A$ must be singular if $det(A^{10})=\textbf{0}.$
\end{proof}
\subsubsection*{Problem 69}
\begin{proof}
    Suppose $A$ a square, skew-symmetric matrix. We know $A^T=-A$. By multiplicativity of determinant, $|A^T|=|-A|= (-1)^n|A|.$ But we know that $|A^T|=|A|$, so $|A|=(-1)^n|A|,$ which satisfies the proof.
\end{proof}
\subsubsection*{Problem 83}
\begin{proof}
    Suppose $A$ is an idempotent matrix and $A^2=A$. By multiplicativity of determinant, $det(A^2)=det(A)$ and $det(A)\times det(A)=det(A).$ If $det(A)\neq 0$, then we can devide booth sides by $det(A)$ and get $det(A)=1$. In case $det(A)=0$, it is clearly obvious that it satisfies $|A^2|=|A|.$ Hence, that satisfies the proof.
\end{proof}
\end{document}