\documentclass[a4paper,12pt]{article}
\usepackage{graphicx, setspace, array, xcolor
, mathtools, contour, tikz, amsthm, setspace, amsmath,amsfonts,amssymb, subcaption, fancyhdr, lipsum,multicol}
\usepackage[a4paper, left=2.5cm, right=2.5cm, top=3cm, bottom=2cm]{geometry}
\usepackage{colortbl}
\contourlength{0.5pt}
\usetikzlibrary{patterns}
\usepackage[english]{babel}
\renewcommand{\qedsymbol}{$\blacksquare$}
\definecolor{darkgreen}{rgb}{0.0, 0.5, 0.0}
\pagestyle{fancy}
\fancyhf{}
\renewcommand{\headrulewidth}{0.4pt}
\fancyhead[L]{\rightmark}
\fancyhead[R]{\thepage}
\title{Linear Algebra}
\author{Assignment 5\\ \\ Yousef A. Abood\\ \\ ID: 900248250}
\date{June 2025}
\setlength{\parindent}{0pt}
\singlespacing
\parskip=1mm
\setcounter{section}{4}
\setcounter{subsection}{0}
\onehalfspacing
\begin{document}
\maketitle
\noindent\makebox[\linewidth]{\rule{15cm}{0.4pt}}
% Section 4.1 Page  159
% 24, 27, 34b
% Section 4.2 Page 166
% 13, 15, 17, 19, 22, 24, 26, 31, 32, 34, 36, 37, 38, 41a
\subsection{Vectors in $R^n$}
\subsubsection*{Problem 24}
$u=(1, 2, 3), v=(2, -2, -1), w=(4, 0, -4).\\
2u+v-w+3z=\textbf{0}
     =(0, 0, 0).$
\begin{align*}
2u+v-w &\iff \textbf{0}-3z\\
       &\iff \textbf{0}+(-3z)\\
       &\iff -3z.\\
&\iff \frac{-1}{3}(2u+v-w)=\frac{-1}{3}(-3z)\\
&\iff \frac{-1}{3}(2u+v-w)=z.\\
&\iff z=\frac{-1}{3}(2(1,2,3)+(2,-2,1)+(-1)(4,0,-4)) \\
&\hspace{1.5cm}= \frac{-1}{3}((2,4,6)+(2,-2,-1)-(4,0,-4))\\
&\hspace{1.5cm}=\frac{-1}{3}(2+2+4,4-2,6-1-4)\\
&\hspace{1.5cm}=\frac{-1}{3}(10,2,1)=(\frac{-10}{3},\frac{-2}{3},\frac{-1}{3}).
\end{align*}
\subsubsection*{Problem 27}
\begin{itemize}
    \item [a)] \textcolor{darkgreen}{It is a scalar multiple of $z$.} $v=\frac{3}{2}z=\frac{3}{2}(3,2,-5)=(\frac{9}{2},3,\frac{-15}{2}).$
    \item [b)] \textcolor{red}{No, it is not a scalar multiple of $z$.} The ratio is different between $\frac{9}{3}=3, \frac{-6}{2}=-3.$
\end{itemize}
\subsubsection*{Problem 34}
\begin{itemize}
    \item [b)] $2w-\frac{1}{2}u=2(2,-2,1,3)-\frac{1}{2}(1,2,-3,1) = (4,-4,2,6)-(\frac{1}{2},1,\frac{-3}{2},\frac{1}{2})=(\frac{7}{2},-5,\frac{7}{2},\frac{11}{2}).$
\end{itemize}
\subsection{Vector Spaces}
\subsubsection*{Problem 13}
\begin{proof}
 Suppose $A,B$ are matrices of size $4 \times 6$. By definition of matrix addition, $A+B=C$ is also of size $4\times 6$, so the addition is closed (1). We also know $A+B=[a_{ij}]+[b_{ij}]=[a_{ij}+b_{ij}]=[b_{ij}+a_{ij}]=B+A,$ clearly commutative (2). Suppose another matrix $D$ of the same size. By associativity of real numbers, $A+B+D=[a_{ij}]+[b_{ij}]+[d_{ij}]=[a_{ij}+b_{ij}+d_{ij}]=[a_{ij}+(b_{ij}+d_{ij})]=[(a_{ij}+b_{ij})+d_{ij}]=A+(B+D)=(A+B)+D,$ we can see addition here is associative (3). We observe that $\textbf{0}_{4\times 6}$ is the additive identity as $A+\textbf{0}_{4\times 6}=A$. (4). The fifth axioms holds as $A-A=A+(-A)=[a_{ij}-a_{ij}]=\textbf{0}_{4\times 6}.$ (5). By definition of matrix scalar multiplication, suppose $c,k \in \mathbb{R}$, and $A,B$ are matrices of size $4 \times 6$, then $(cA)$ is a matrix of size $4 \times 6$ and scalar multiplication is closed (6), it also implies that $c(A+B)=c[a_{ij}+b_{ij}]=[ca_{ij}+cb_{ij}]=[ca_{ij}]+[cb_{ij}]=cA+cB$ (7), and $(c+k)A=(c+k)[a_{ij}]=[(c+k)a_{ij}]=[ca_{ij}+ka_{ij}]=cA+kA$ (8). We also have $cdA=c(dA)=(cd)A$ by scalar multiplication of matrices (9). It is obvious that $1A=A$ (10). Hence, \textcolor{darkgreen}{It is a vector space.}   
\end{proof}
\subsubsection*{Problem 15}
It is not a vector space as $x^3+5x^2+(-1)(x^3)=x^3+5x^2-x^3=5x^2$, clearly not closed under addition as we can see the result is a polynomial of degree 2.
\subsubsection*{Problem 17}
It is not a vector space, take $p(x)=x$ and take $q(x)=-x$, then $p(x)+q(x)=x-x=0$ wich is not in the set. Hence, the set is not closed under addition.
\subsubsection*{Problem 19}
As stated in the lecture notes, the set $\mathcal{P}_n$ of all polynomials of degree n or less is a vector space under polynomial addition and scalar multiplication. Therefore, the set of all polynomials of degree four or less is a vector space.
\subsubsection*{Problem 22}
It is not a vector space, take the pair $(1,1)$. Observe $(-1)(1,1)=(-1,-1)$, clearly not closed under scalar multiplication.
\subsubsection*{Problem 24}
\textcolor{darkgreen}{It is a vector space.}\\
(1),(2) Choose $x$ to be $a,b$, where $a,b \in \mathbb{R}$. We observe $(a, \frac{1}{2}a)+(b,\frac{1}{2}b)=(a+b, \frac{1}{2}a+\frac{1}{2}b)=(a+b, \frac{1}{2}(a+b))=(b+a, \frac{1}{2}(b+a))$, clearly closed under addition and it satisfies the commutativity of addition.\\
(3) Choose $a,b,c$, where $a,b,c \in \mathbb{R}.$ Then $(a,\frac{1}{2}a)+(b,\frac{1}{2}b)+(c,\frac{1}{2}c)=(a,\frac{1}{2}a)+(b+c,\frac{1}{2}(b+c))=(a+b,\frac{1}{2}(a+b))+(c,\frac{1}{2}c)$, satisfies the addition associativity.\\
(4),(5) It is clear that $(0,\frac{1}{2} (0))=(0,0)$ is the additive identity, and $(-a,\frac{1}{2}(-a))$ is the additive inverse, where $a\in \mathbb{R}$. \\
(6),(7),(8) Choose $k,d$ to be real numbers scalars. See that $c(a,\frac{1}{2})=(ca,\frac{c}{2}a)=(ca,\frac{1}{2}(ca)),$ and $c((a,\frac{1}{2}a)+(b,\frac{1}{2}b))=c(a+b,\frac{1}{2}(a+b))=(c(a+b),\frac{1}{2}c(a+b))=(ca+cb,\frac{1}{2}(ca+cb))=(ca,\frac{c}{2}a)+(cb,\frac{c}{2}b).$
We can also see $(c+d)(a,\frac{1}{2}a)=((c+d)a,\frac{(c+d)}{2}a)=(ca+da,\frac{c}{2}a+\frac{d}{2}a)=(ca,\frac{c}{2}a)+(da,\frac{d}{2}a).$\\
(9) It is clear that $(cd)(a,\frac{1}{2}a)=c(da,\frac{d}{2}a)$, and $1(a,\frac{1}{2}a)=(a,\frac{1}{2}a).$
\subsubsection*{Problem 26}
\textcolor{red}{It is not a vector space.}\\ Suppose $A$ is a matrix of the form $\begin{bmatrix}
   a& b \\
   c& 1
\end{bmatrix}$, we get a matrix $\begin{bmatrix}
  a+a & b+b \\
   c+c & 2
\end{bmatrix}$ when we add $A$ to itself, clearly it is not closed ander addition.
\subsubsection*{Problem 31}
\textcolor{red}{It is not a vector space.}\\ Choose two singular matrices $\begin{bmatrix}
  1 & 0 \\
  0 & 0
\end{bmatrix}, \begin{bmatrix}
  0 & 0 \\
  0 & 1
\end{bmatrix}$. Adding them, we get $\begin{bmatrix}
  1 & 0 \\
  0 & 1
\end{bmatrix}$, which is the identity matrix, and we know the identity matrix is invertible, clearly not closed under addition.
\subsubsection*{Problem 32}
\textcolor{red}{It is not a vector space.}\\ Choose two invertible matrices $\begin{bmatrix}
  1 & 0 \\
  0 & 1
\end{bmatrix}, \begin{bmatrix}
  -1 & 0 \\
  0 & 1
\end{bmatrix}$. Adding them, we get $\begin{bmatrix}
  0 & 0 \\
  0 & 2
\end{bmatrix}$, which is a singular matrix, clearly not closed under addition.

\subsubsection*{Problem 34}
\textcolor{darkgreen}{It is a vector space.}\\
Suppose $A,B$ are upper triangular square matrices of size 3, and $c,k$ are real numbers.\\
(1),(2),(6) From matrix addition and scalar multiplication, we know $A+B$ and $cA$ are upper triangular square matrices of size 3, clearly closed under addition and scalar multiplication. We also know $A+B=B+A$, which satisfies addition commutativity.\\
(3),(7) Suppose $D$ an upper triangular square matrix of size 3. By matrix addtion, we know $(A+B)+D=A+(B+D)$. We also know that $c(A+B)=cA+cB.$\\
(4),(5),(10) We clearly see that the zero square matrix of size 3 is an additive identity in this set, and the additive inverse of any matrix $A$ is $-A$ such that $A+(-A)=\textbf{0}_{3\times 3}$. $1A=A$ is obviously satisfied for any matrix.\\
(8) We observe $(c+k)=A=(c+k)[a_{ij}]=[(c+k)a_{ij}]=[ca_{ij}+ka_{ij}]=c[a_{ij}]+k[a_{ij}]=cA+kA.$\\
(9) Finally, observe $(ck)A=ck[a_{ij}]=c[ka_{ij}]=c(kA).$
\subsubsection*{Problem 36}
\textcolor{darkgreen}{It is a vector space.}\\
Suppose $f(x),g(x)$ to be a continous functions defined on the interval [-1,1]. It is clear that this set is a subset of the set of continous real valued functions, which we proved it is a vector space in class. 
Since booth $f,g$ are defined on the interval, and by what we defined in class, $(f+g)(x)=f(x)+g(x)$ and $(cf)(x)=c(f(x)).$ It is clear that it is closed under addition and scalar multiplication. It is also non empty as this set includes the zero function. So, it passes the subspace test. Hence, the set of all continous functions defined on interval $[-1,1]$ is a vector space. \\

\subsubsection*{Problem 37}

\subsubsection*{Problem 38}
\subsubsection*{Problem 41}
\begin{itemize}
    \item [a)]
\end{itemize}
\end{document}