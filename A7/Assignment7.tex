\documentclass[a4paper,12pt]{article}
\usepackage{graphicx, setspace, array, xcolor, mathtools, contour, tikz, amsthm, setspace, amsmath,amsfonts,amssymb, subcaption, fancyhdr, lipsum,multicol}
\usepackage{colortbl}
\contourlength{0.5pt}
\usetikzlibrary{patterns}
\usepackage[english]{babel}
\renewcommand{\qedsymbol}{$\blacksquare$}
\definecolor{darkgreen}{rgb}{0.0, 0.5, 0.0}
\pagestyle{fancy}
\fancyhf{}
\renewcommand{\headrulewidth}{0.4pt}
\fancyhead[L]{\rightmark}
\fancyhead[R]{\thepage}
\title{Linear Algebra}
\author{Assignment 7\\ \\ Yousef A. Abood\\ \\ ID: 900248250}
\date{June 2025}
\setlength{\parindent}{0pt}
\singlespacing
\parskip=1mm
\setcounter{section}{4}
\setcounter{subsection}{3}
\onehalfspacing
\begin{document}
\maketitle
\noindent\makebox[\linewidth]{\rule{15cm}{0.4pt}}

\subsection{Spanning Sets and Linear Independence}

\subsubsection*{Problem 2}
\begin{itemize}
    \item [d)] \begin{align*}
        &(1, -5, -5)= a(1,2,-2)+b(2,-1,1)\\
        \iff& 1=a+2b, -5=2a-b,-5= -2a+b
    \end{align*}
We solve this system of equations:
\begin{align*}
    \begin{bmatrix}
        1&2&1\\
        2&-1&-5\\
        -2&1&-5
    \end{bmatrix} \xrightarrow {{(R_2-2R_1)}\to{R_2}}& \begin{bmatrix}
        1&2&1\\
        0&-5&-7\\
        -2&1&-5
    \end{bmatrix} \xrightarrow{{(R_3+2R_1)}\to{R_3}} \begin{bmatrix}
        1&2&1\\
        0&-5&-7\\
        0&5&-3
    \end{bmatrix}\\ \xrightarrow  {{(\frac{-1}{5})R_2}\to{R_2}}& \begin{bmatrix}
        1&2&1\\
        0&1&\frac{7}{5}\\
        0&5&-3
    \end{bmatrix} \xrightarrow{{(R_3-5R_2)}\to{R_3}} \begin{bmatrix}
        1&2&1\\
        0&1&\frac{7}{5}\\
        0&0&-10
    \end{bmatrix}.
\end{align*}
From the last step we get $0a+0b=10.$ So the system has no solution and the vector $u$ cannot be written as linear combination of vectors in $S.$
\end{itemize}
\subsubsection*{Problem 23}
Let $\vec{v}=(a,b,c)\in R^3$.\\ We observe if $x(1,-2,0)+y(0,0,1)+z(-1,2,0)=(a,b,c)$ has a unique solution:
\begin{align*}
\begin{bmatrix}
    1&0&-1\\
    -2&0&2\\
    0&1&0\\
\end{bmatrix} \xrightarrow{{(R_2+2R_1)}\to{R_2}} \begin{bmatrix}
  1 & 0 & -1 \\
  0 & 0 & 0 \\
  0 & 1 & 0
\end{bmatrix}
\end{align*}
It has a row of zeros, so it is not invertible and it does not have a unique solution, so this set does not span $R^3.$
\subsubsection*{Problem 46}
We see:
\[2+3x+x^2=6+5x+x^2+2(-2-x).\]
We could form one of the vectors by a linear combination of the other vector is the same set. So the set is linearly dependent.
\subsubsection*{Problem 50}
We see that every $1$ in each matrix corresponds to a zero in the other matrices. Hence, it is impossible to obtain one matrix by a linear combination by the other matrices. So the set is linearly independent.
\subsubsection*{Problem 57}
\begin{itemize}
    \item [b)] By the equation 
\[a(t,1,1)+b(1,0,1)+c(1,1,3t)=(0,0,0)\]
We construct this system of equations:
\[at+b+c=0,\\ a+c=0,\\ a+b+3tc=0.\]
We find the values of $t$ that makes the following matrix invertible.
\begin{align*}
    \begin{vmatrix}
      t & 1 & 1 \\
      1 & 0 & 1 \\
      1 & 1 & 3t
    \end{vmatrix} = t(-1)-(3t-1)+(1)=-4t+2.
\end{align*}
We see the determinant is not zero when $-4t+2 \neq 0 \iff -4t\neq -2 \iff t \neq \frac{1}{2}.$
So every other value for $t$ makes this set linearly independent.
\end{itemize}
\subsubsection*{Problem 62}
Attached in the mail (Written in paper.)
\subsubsection*{Problem 73}
% Solution here
\begin{proof}
    Since $S$ is linearly Independent, $au+bv=\vec{0}$ has only the trivial solution. Pick two scalars $x,y \in \mathbb{R}-\{0\}.$ Assume $x(u+v)+y(u-v)=(0,0)$ has other solution than the trivial one. Observe $xu+xv+yu-yv=(0,0)=(x+y)u+(x-y)v.$ But we know ${u,v}$ are linearly independent, so $x+y=0, x-y=0$ and clearly $x=0,y=0.$ which is a contradiction. Hence, the set $\{u+v, u-v\}$ is linearly independent.
\end{proof}
\subsubsection*{Problem 75}
\begin{proof}
    Let $A$ be a nonsingular matrix of order 3, so $A^{-1}$ exists. Pick scalars $a,b,c \in \mathbb{R}$. We want to prove $aAv_1+bAv_2+cAv_3=\vec{0}.$ only has the trivial solution. Observe that we can multiply both sides by $A^{-1}$ and get $aA^{-1}Av_1+bA^{-1}Av_2+cA^{-1}Av_3=A^{-1}\textbf{0} = av_1+bv_2+cv_3=\textbf{0}.$ But we know $\{v_1,v_2,v_3\}$ is linearly independent, so $a=0,b=0,c=0$ and $\{Av_1,Av_2,Av_3\}$ is also linearly independent.
\end{proof}
If $A$ is a singular, the set is not linearly independent. A counter example is the zero matrix $\textbf{0}_{3\times 3}$, which will satisfy the equation for any scalars $a,b,c \in \mathbb{R}$, so the equation will have other solutions than the trivial one and the set is linearly dependent.
\end{document}