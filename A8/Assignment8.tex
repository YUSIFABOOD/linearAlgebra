\documentclass[a4paper,12pt]{article}
\usepackage{graphicx, setspace, array, xcolor, mathtools, contour, tikz, amsthm, setspace, amsmath,amsfonts,amssymb, subcaption, fancyhdr, lipsum,multicol}
\usepackage{colortbl}
\contourlength{0.5pt}
\usetikzlibrary{patterns}
\usepackage[english]{babel}
\renewcommand{\qedsymbol}{$\blacksquare$}
\definecolor{darkgreen}{rgb}{0.0, 0.5, 0.0}
\pagestyle{fancy}
\fancyhf{}
\renewcommand{\headrulewidth}{0.4pt}
\fancyhead[L]{\rightmark}
\fancyhead[R]{\thepage}
\title{Linear Algebra}
\author{Assignment 8\\ \\ Yousef A. Abood\\ \\ ID: 900248250}
\date{June 2025}
\setlength{\parindent}{0pt}
\singlespacing
\parskip=1mm
\setcounter{section}{4}
\setcounter{subsection}{4}
\onehalfspacing
\begin{document}
\maketitle
\noindent\makebox[\linewidth]{\rule{15cm}{0.4pt}}

\subsection{Basis and Dimension}
\subsubsection*{Problem 20}
\begin{proof}
    We know that the basis must be linearly independent. But we see $(-1,0,0)=0(1,0,0)+(-1)(1,0,0)$, clearly the set is not linearly independent. Therefore, we proved the set is not a basis for $R^3.$
\end{proof}
\subsubsection*{Problem 30}
\begin{proof}
    We know that the basis must be linearly independent. But we see $3x^2=3(1-2x-x^2)+(-3+6x)$, clearly the set is not linearly independent. Therefore, we proved the set is not a basis for $P_2.$
\end{proof}
\subsubsection*{Problem 34}
\begin{proof}
    We know that the basis of any $\mathbb{M}_{m\times n}$ matrix is a set contains $m \times n$ many matrices of size $m \times n$. So the basis for $\mathbb{M}_{2\times 2}$ will have $2 \times 2 =4$ elements, but the set contains only three elements. Therefore, it is not a basis for $\mathbb{M}_{2\times 2}.$
\end{proof}
\subsubsection*{Problem 44}
\begin{proof}
    We know that the basis of any $R^n$ vector space is a set contains $n$ many vectors of size $n$ and it must be linearly independent. The given set, $S={(0,0,0),(1,5,6),(6,2,1)}$, which has three elements but it has the zero vector, so it is linearly dependent and it cannot be the basis for $R^3$. Therefore, we proved $S$ is not a basis for $R^3.$
\end{proof}
\subsubsection*{Problem 48}
\begin{proof}
First, we need to determine whether the set $S$ is a spanning set. Let $(a,b,c,d)$ any vector that contains the coffectients of any polynomial in $P_3.$ Pick scalars $x,y,z,w \in \mathbb{R}$. We need to show that $x(0,4,0,-1,0)+y(5,0,0,1)+z(1,3,0,0)+w(0,0,-3,2)=(a,b,c,d)$ has a unique solution $(x,y,z,w)$ for every choice of $(a,b,c,d)$. We get the system: 
\begin{align*}
    5y + z         &= a \\
    4x + 3z        &= b \\
    -3w            &= c \\
    y + 2w         &= d
\end{align*}
We construct the matrix of the coffectients and compute its determinant:
\begin{align*}
    A=\begin{bmatrix}
        0&5&1&0\\
        4&0&3&0\\
        0&0&0&-3\\
        0&1&0&2
    \end{bmatrix},\end{align*} \begin{align*} det(A)=-4\begin{vmatrix}
        5&1&0\\
        0&0&-3\\
        1&0&2
    \end{vmatrix} = -4* 3 (5*0-1)=12 \neq 0.
\end{align*}
So the matrix is invertible and the system has a unique solution. Hence, the set is a spanning set of $P_3. (1)$
If we choose $(a,b,c,d)$ to be $(0,0,0,0)$, and we know from $(1)$ that the coffectient matrix is invertible and the system has a unique solution for every choice of $(a,b,c,d).$ Then the trivial solution is the only solution. Hence, the set is linearly Independent. (2)
Therefore, from (1),(2), we proved that $S$ is a basis for $P_3.$ 
\end{proof}
\subsubsection*{Problem 70}
We know that a basis for $R^3$ must has three vectors. We find one example of that vector: $(0,0,-2)$. To see that it is a basis we show it is a linearly independent spanning set. Pick scalars $x,y,z \in \mathbb{R}.$ We need to show that $x(1,0,2)+y(0,1,1)+z(0,0,-2)=(a,b,c)$ has a unique solution.
We get the system:
\begin{align*}
    x+2z=a\\
    y+z=b\\
    -2z=c
\end{align*}
We compute the determinant of the coffectient matrix of this system:
\begin{align*}
    \begin{vmatrix}
        1&0&2\\
        0&1&1\\
        0&0&-2
    \end{vmatrix} = -2(1-0)=-2 \neq 0.
\end{align*}
So the matrix is invertible and the system has a unique solution. If we choose $(a,b,c)=(0,0,0)$ it will also have one solution -as we proved the coffectient matrix is invertible- which is the trivial one. Therefore, we found a basis for $R^3.$
\subsubsection*{Problem 71}
\begin{itemize}
    \item [a)] The line represented by the equation $y=\frac{1}{2}x.$
    \item [b)] A basis for $W$ is the set ${(2,1)}.$ \\ For every $(2t,t)$, we can represent it by $t(2,1).$ Clearly the set spans $W.$ The set has one vector so it is linearly Independent.
    
    \item [c)] Since the basis consists of one vector, the dimension of $W$ is 1.
\end{itemize}
\subsubsection*{Problem 78}
\begin{itemize}
    \item [a)] A valid choice of a basis is ${(1,0,1,2),(4,1,0,-1)}$. \\ We see it spans $W$ as we can construct any vector in $W$ by: $s(1,0,1,2)+t(4,1,0,-1)=(s+4t,t,s,2s-t).$ It is obviously linearly independent as if we multiplied the first vector by any scalar we cannot get the second vector as we have zero in the first vector that corresponds to 1 in the second vector.
    \item [b)] Since the basis consists of two vectors, the dimension of $W$ is 2.
\end{itemize}
\end{document}