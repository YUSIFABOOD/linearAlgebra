\documentclass[a4paper,12pt]{article}
\usepackage{graphicx, setspace, array, xcolor, mathtools, contour, tikz, amsthm, setspace, amsmath,amsfonts,amssymb, subcaption, fancyhdr, lipsum,multicol, geometry}
\usepackage{colortbl}
\geometry{a4paper, margin=1in}
\contourlength{0.5pt}
\usetikzlibrary{patterns}
\usepackage[english]{babel}
\renewcommand{\qedsymbol}{$\blacksquare$}
\definecolor{darkgreen}{rgb}{0.0, 0.5, 0.0}
\pagestyle{fancy}
\fancyhf{}
\renewcommand{\headrulewidth}{0.4pt}
\fancyhead[L]{\rightmark}
\fancyhead[R]{\thepage}
\title{Linear Algebra}
\author{Assignment 9\\ \\ Yousef A. Abood\\ \\ ID: 900248250}
\date{June 2025}
\setlength{\parindent}{0pt}
\singlespacing
\parskip=1mm
\setcounter{section}{4}
\setcounter{subsection}{5}
\onehalfspacing
\begin{document}
\maketitle
\noindent\makebox[\linewidth]{\rule{15cm}{0.4pt}}
\subsection{Rank of a matrix}
\subsubsection*{Problem 11}
\begin{itemize}
    \item [a)]\begin{align*}
    \begin{bmatrix}
      -2 & -4 & 4 &5\\
      3 & 6 & -6 &-4\\
      -2 & -4 & 4 &9
    \end{bmatrix} \xrightarrow{{\frac{-1}{2}R_1}\to{R_1}}&
    \begin{bmatrix}
      1 & 2 & -2 &\frac{-5}{2}\\
      3 & 6 & -6 &-4\\
      -2 & -4 & 4 &9
    \end{bmatrix}\\ \xrightarrow{{(R_2-3R_1)}\to{R_2}}& \begin{bmatrix}
      1 & 2 & -2 &\frac{-5}{2}\\
      0 & 0 & 0 &\frac{7}{2}\\
      -2 & -4 & 4 &9
    \end{bmatrix}\\ \xrightarrow{{(R_3+2R_1)}\to{R_3}}& \begin{bmatrix}
      1 & 2 & -2 &\frac{-5}{2}\\
      0 & 0 & 0 &\frac{7}{2}\\
      0 & 0 & 0 &4
    \end{bmatrix}\\ \xrightarrow{{\frac{7}{2}R_2}\to{R_2}}& 
    \begin{bmatrix}
      1 & 2 & -2 &\frac{-5}{2}\\
      0 & 0 & 0 &1\\
      0 & 0 & 0 &4
    \end{bmatrix}\\ \xrightarrow{{(R_3-4R_2)}\to{R_3}}&
    \begin{bmatrix}
      1 & 2 & -2 &\frac{-5}{2}\\
      0 & 0 & 0 &1\\
      0 & 0 & 0 &0
    \end{bmatrix}.
\end{align*}
Thus, the basis for the row space is the set $\{(1,2,-2,\frac{-5}{2}),(0,0,0,1)\}$.
    \item[b)] The rank of the matrix is the cardinality of basis for the row space. So, the rank is $2.$
\end{itemize}

\subsubsection*{Problem 14}
We obtain a matrix $A$ from the vectors in $S$. \[A=\begin{bmatrix}
  1 & 2 & 4 \\
  -1 & 3 & 4 \\
  2 & 3 & 1
\end{bmatrix}\]
We keep applying the ERO till we reach the REF.
\begin{align*}
    \begin{bmatrix}
  1 & 2 & 4 \\
  -1 & 3 & 4 \\
  2 & 3 & 1
\end{bmatrix} \xrightarrow{{(R_2+R_1)}\to{R_2}}&
 \begin{bmatrix}
  1 & 2 & 4 \\
  0 & 5 & 8 \\
  2 & 3 & 1
\end{bmatrix} \xrightarrow{{(R_3-2R_1)}\to{R_3}}
\begin{bmatrix}
  1 & 2 & 4 \\
  0 & 5 & 8 \\
  0 & -1 & -7
\end{bmatrix}\\ \xrightarrow{{\frac{1}{5} R_2}\to{R_2}}&
\begin{bmatrix}
  1 & 2 & 4 \\
  0 & 1 & \frac{8}{5} \\
  0 & -1 & -7
\end{bmatrix} \xrightarrow{{(R_3+R_2)}\to{R_3}}
\begin{bmatrix}
  1 & 2 & 4 \\
  0 & 1 & \frac{8}{5} \\
  0 & 0 & \frac{-27}{5}
\end{bmatrix}\\ \xrightarrow{{(\frac{-5}{27})R_3}\to{R_3}}&
\begin{bmatrix}
  1 & 2 & 4 \\
  0 & 1 & \frac{8}{5} \\
  0 & 0 & 1
\end{bmatrix}
\end{align*}
Since the rows of the matrix are exactly the vectors in $S$, we get that the supspace of $R^3= Span(S)= R(A).$
Therefore, the set $\{(1,2,4),(0,1,\frac{8}{5}), (0,0,1)\}$ forms a basis for the subspace.
\subsubsection*{Problem 25}
\begin{itemize}
\item[a)] \begin{align*}
    \begin{bmatrix}
      2 & 4 & -3 &-6\\
      7 & 14 & -6 &-3\\
      -2 & -4 & 1 &-2\\
      2&4&-2&-2
    \end{bmatrix} \xrightarrow{{\frac{1}{2}R_1}\to{R_1}}&
    \begin{bmatrix}
      1 & 2 & \frac{-3}{2} &-3\\
      7 & 14 & -6 &-3\\
      -2 & -4 & 1 &-2\\
      2&4&-2&-2
    \end{bmatrix} \xrightarrow{{(R_2-7R_1)}\to{R_2}}
    \begin{bmatrix}
      1 & 2 & \frac{-3}{2} &-3\\
      0 & 0 & \frac{9}{2} &18\\
      -2 & -4 & 1 &-2\\
      2&4&-2&-2
    \end{bmatrix}\\ \xrightarrow{{(R_3+2R_1)}\to{R_3}}&
    \begin{bmatrix}
      1 & 2 & \frac{-3}{2} &-3\\
      0 & 0 & \frac{9}{2} &18\\
      0 & 0 & -2 &-8\\
      2&4&-2&-2
    \end{bmatrix} \xrightarrow{{(R_4-2R_1)}\to{R_4}}
    \begin{bmatrix}
      1 & 2 & \frac{-3}{2} &-3\\
      0 & 0 & \frac{9}{2} &18\\
      0 & 0 & -2 &-8\\
      0&0&-1&4
    \end{bmatrix} \\ \xrightarrow{{\frac{2}{9}}R_2\to{R_2}}&
    \begin{bmatrix}
      1 & 2 & \frac{-3}{2} &-3\\
      0 & 0 & 1 &4\\
      0 & 0 & -2 &-8\\
      0&0&-1&4
    \end{bmatrix} \xrightarrow{{(R_3+2R_2)}\to{R_3}}
    \begin{bmatrix}
      1 & 2 & \frac{-3}{2} &-3\\
      0 & 0 & 1 &4\\
      0 & 0 & 0 &0\\
      0&0&-1&4
    \end{bmatrix}\\ \xrightarrow{{R_3}\leftrightarrow{R_4}}&
    \begin{bmatrix}
      1 & 2 & \frac{-3}{2} &-3\\
      0 & 0 & 1 &4\\
      0&0&-1&4\\
      0 & 0 & 0 &0
    \end{bmatrix} \xrightarrow{{(R_3+R_2)}\to{R_3}}
    \begin{bmatrix}
      1 & 2 & \frac{-3}{2} &-3\\
      0 & 0 & 1 &4\\
      0&0&0&8\\
      0 & 0 & 0 &0
    \end{bmatrix}\\ \xrightarrow{{\frac{1}{8}R_3}\to{R_3}}&
    \begin{bmatrix}
      1 & 2 & \frac{-3}{2} &-3\\
      0 & 0 & 1 &4\\
      0&0&0&1\\
      0 & 0 & 0 &0
    \end{bmatrix}
\end{align*}
Observe that the pivot columns of the REF matrix are the $1^{st}$, $2^{nd}$, and the $4^{th}$ column. Consequently, the set contains the $1^{st}$, the $2^{nd}$, and the $4^{th}$ columnns of matrix $A$:
\[
\left\{
  \begin{bmatrix}
    2 \\
    7 \\
    -2 \\
    2
  \end{bmatrix},
  \begin{bmatrix}
    -3\\
    -6\\
    1\\
    -2
  \end{bmatrix},
  \begin{bmatrix}
    -6\\
    -3\\
    -2\\
    -2
  \end{bmatrix}
\right\}
\]
are the basis of the column space of $A$.
\item[b)] The rank of the matrix $A$ = $dim(C(A))= 3.$
\end{itemize}
\subsubsection*{Problem 35}
We keep applying the EROs till we reach the REF:
\begin{align*}
  A=\begin{bmatrix}
    5 & 2 \\
    3 & -1 \\
    2 & 1
  \end{bmatrix} \xrightarrow{{\frac{1}{5}R_1}\to{R_1}}&
  \begin{bmatrix}
    1 & \frac{2}{5} \\
    3 & -1 \\
    2 & 1
  \end{bmatrix} \xrightarrow[(R_3-2R_1)\to R_3]{(R_2-3R_1)\to R_2}
  \begin{bmatrix}
    1 & \frac{2}{5} \\
    0 & \frac{-11}{5}  \\
    0 & \frac{1}{5}
  \end{bmatrix}\\ \xrightarrow{{\frac{-5}{11}R_2}\to{R_2}}&
  \begin{bmatrix}
    1 & \frac{2}{5} \\
    0 & 1  \\
    0 & \frac{1}{5}
  \end{bmatrix} \xrightarrow{{(R_3-\frac{1}{5}R_2)}\to{R_3}}
  \begin{bmatrix}
    1 & \frac{2}{5} \\
    0 & 1  \\
    0 & 0
  \end{bmatrix} =B
\end{align*}
We know that $N(B)=N(A)$. And we get the system \[x+\frac{2}{5}y=0\\ y=0.\]
This system clearly only has the trivial solution $(0,0)$. Hence, $N(A)={(0,0)}$ 
\subsubsection*{Problem 42}
Since we know that $A,B$ are row equivalent, we can obtain $B$ by applying EROs on $A$ and they have the same row space, rank, nullity.
\begin{itemize}
  \item [a)] We can see that $B$ is in the REF, and we know that the rank is the number of the pivot columnns and the nullity is the number of the non-pivot columns. It is clear that the $1^{st},2^{nd},4^{th}$ columns are pivot columns and the $3^{rd},5^{th}$ columns are the non-pivot columns. Hence, $rank(A)=3, nullity(A)=2$
  \item [b)] Since $B$ is in REF we obtain the follwoing homogeneous system of linear equations 
  \begin{align*}
    x+z+l=0 \\ y-2z+3l=0\\ w-5l=0
  \end{align*}
  , the non-pivot columns corresponds to free variables, so we choose $z=s, l=t$ and get that $w=5t, y=2s-3t, x= -s-t.$ and that is the solution of the homogeneous system.
  So we get that $N(A)=\left\{
  \begin{bmatrix}
    -s-t \\
    2s-3t\\
    s \\
    5t\\
    t
  \end{bmatrix}: s,t \in \mathbb{R}
\right\}$
and for every vector $\vec{v} \in N(A)$ we have:
\[\vec{v}= \begin{bmatrix}
    -s-t \\
    2s-3t\\
    s \\
    5t\\
    t
  \end{bmatrix}=\begin{bmatrix}
    -s \\
    2s\\
    s \\
    0\\
    0
  \end{bmatrix}+\begin{bmatrix}
    -t \\
    -3t\\
     0\\
    5t\\
    t
  \end{bmatrix}= s\begin{bmatrix}
    -1 \\
    2\\
    1 \\
    0\\
    0
  \end{bmatrix}+t\begin{bmatrix}
    -1 \\
    -3\\
     0\\
    5\\
    1
  \end{bmatrix}\]
  Thus, every vector $\vec{v} \in N(A)$ can be obtained by a linear combination of vectors in the set
  \[S=\left\{\begin{bmatrix}
    -1 \\
    2\\
    1 \\
    0\\
    0
  \end{bmatrix},\begin{bmatrix}
    -1 \\
    -3\\
     0\\
    5\\
    1
  \end{bmatrix}\right\}\]
  So it is clearly a spanning set, it is also linearly independent as we have zeros in one vector that correspnds to numbers in the others, whihc implies we cannot get one by a linear combination of the other. Therefore, $S$ is a basis of $N(A)$
  \item [c)] The basis of the row space is the set that contains the pivot rows of the REF matrix. Hence, the basis $T=\{(1,0,1,0,1), (0,1,-2,0,3),(0,0,0,1,-5)\}$.
  \item [d)] The basis of the column space of $A$ is the set that contains the columns in $A$ that correspnds to the pivot columns in $B$. Hence, the basis \[W=\{(-2,1,3,1),(-5,3,11,7),(0,1,7,5)\}.\]
  \item [e)] Since $dim(R(A))=3$, and the number of rows in $A$ is 4, then the rows of $A$ are not linearly independent.
  \item [f)]\begin{itemize}
    \item [i)] We know that this set is the basis of $C(A)$ from (d). And we know that a basis is linearly independent. Hence, the set is linearly independent.
    \item [ii)] The set is linearly dependent as we can get $a_{3}=a_1-a_2.$
    \item [iii)] Since we have zero in $a_3,a_1$ that correspnds to number in $a_5$, then we cannot get any of them as linear combination of the others. Hence, the set is linearly independent.
  \end{itemize}
\end{itemize}
\end{document}