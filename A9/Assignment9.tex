\documentclass[a4paper,12pt]{article}
\usepackage{graphicx, setspace, array, xcolor, mathtools, contour, tikz, amsthm, setspace, amsmath,amsfonts,amssymb, subcaption, fancyhdr, lipsum,multicol, geometry}
\usepackage{colortbl}
\geometry{a4paper, margin=1in}
\contourlength{0.5pt}
\usetikzlibrary{patterns}
\usepackage[english]{babel}
\renewcommand{\qedsymbol}{$\blacksquare$}
\definecolor{darkgreen}{rgb}{0.0, 0.5, 0.0}
\pagestyle{fancy}
\fancyhf{}
\renewcommand{\headrulewidth}{0.4pt}
\fancyhead[L]{\rightmark}
\fancyhead[R]{\thepage}
\title{Linear Algebra}
\author{Assignment 9\\ \\ Yousef A. Abood\\ \\ ID: 900248250}
\date{June 2025}
\setlength{\parindent}{0pt}
\singlespacing
\parskip=1mm
\setcounter{section}{4}
\setcounter{subsection}{5}
\onehalfspacing
\begin{document}
\maketitle
\noindent\makebox[\linewidth]{\rule{15cm}{0.4pt}}
\subsection{Rank of a matrix}
\subsubsection*{Problem 11}
\begin{itemize}
    \item [a)]\begin{align*}
    \begin{bmatrix}
      -2 & -4 & 4 &5\\
      3 & 6 & -6 &-4\\
      -2 & -4 & 4 &9
    \end{bmatrix} \xrightarrow{{\frac{-1}{2}R_1}\to{R_1}}&
    \begin{bmatrix}
      1 & 2 & -2 &\frac{-5}{2}\\
      3 & 6 & -6 &-4\\
      -2 & -4 & 4 &9
    \end{bmatrix}\\ \xrightarrow{{(R_2-3R_1)}\to{R_2}}& \begin{bmatrix}
      1 & 2 & -2 &\frac{-5}{2}\\
      0 & 0 & 0 &\frac{7}{2}\\
      -2 & -4 & 4 &9
    \end{bmatrix}\\ \xrightarrow{{(R_3+2R_1)}\to{R_3}}& \begin{bmatrix}
      1 & 2 & -2 &\frac{-5}{2}\\
      0 & 0 & 0 &\frac{7}{2}\\
      0 & 0 & 0 &4
    \end{bmatrix}\\ \xrightarrow{{\frac{7}{2}R_2}\to{R_2}}& 
    \begin{bmatrix}
      1 & 2 & -2 &\frac{-5}{2}\\
      0 & 0 & 0 &1\\
      0 & 0 & 0 &4
    \end{bmatrix}\\ \xrightarrow{{(R_3-4R_2)}\to{R_3}}&
    \begin{bmatrix}
      1 & 2 & -2 &\frac{-5}{2}\\
      0 & 0 & 0 &1\\
      0 & 0 & 0 &0
    \end{bmatrix}.
\end{align*}
Thus, the basis for the row space is the set $\{(1,2,-2,\frac{-5}{2}),(0,0,0,1)\}$.
    \item[b)] The rank of the matrix is the cardinality of basis for the row space. So, the rank is $2.$
\end{itemize}

\subsubsection*{Problem 14}
We obtain a matrix $A$ from the vectors in $S$. \[A=\begin{bmatrix}
  1 & 2 & 4 \\
  -1 & 3 & 4 \\
  2 & 3 & 1
\end{bmatrix}\]
We keep applying the ERO till we reach the REF.
\begin{align*}
    \begin{bmatrix}
  1 & 2 & 4 \\
  -1 & 3 & 4 \\
  2 & 3 & 1
\end{bmatrix} \xrightarrow{{(R_2+R_1)}\to{R_2}}&
 \begin{bmatrix}
  1 & 2 & 4 \\
  0 & 5 & 8 \\
  2 & 3 & 1
\end{bmatrix} \xrightarrow{{(R_3-2R_1)}\to{R_3}}
\begin{bmatrix}
  1 & 2 & 4 \\
  0 & 5 & 8 \\
  0 & -1 & -7
\end{bmatrix}\\ \xrightarrow{{\frac{1}{5} R_2}\to{R_2}}&
\begin{bmatrix}
  1 & 2 & 4 \\
  0 & 1 & \frac{8}{5} \\
  0 & -1 & -7
\end{bmatrix} \xrightarrow{{(R_3+R_2)}\to{R_3}}
\begin{bmatrix}
  1 & 2 & 4 \\
  0 & 1 & \frac{8}{5} \\
  0 & 0 & \frac{-27}{5}
\end{bmatrix}\\ \xrightarrow{{(\frac{-5}{27})R_3}\to{R_3}}&
\begin{bmatrix}
  1 & 2 & 4 \\
  0 & 1 & \frac{8}{5} \\
  0 & 0 & 1
\end{bmatrix}
\end{align*}
Since the rows of the matrix are exactly the vectors in $S$, we get that the supspace of $R^3= Span(S)= R(A).$
Therefore, the set $\{(1,2,4),(0,1,\frac{8}{5}), (0,0,1)\}$ forms a basis for the subspace.
\subsubsection*{Problem 25}
\subsubsection*{Problem 35}
\subsubsection*{Problem 42}

\end{document}