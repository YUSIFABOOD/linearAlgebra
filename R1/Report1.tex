\documentclass[a4paper,12pt]{article}
\usepackage{graphicx, setspace, array, xcolor
, mathtools, contour, tikz, amsthm, setspace, amsmath,amsfonts,amssymb, subcaption, fancyhdr, lipsum,multicol, amsthm}
\usepackage[a4paper, left=2.5cm, right=2.5cm, top=3cm, bottom=2cm]{geometry}
\usepackage{colortbl}
\contourlength{0.5pt}
\usetikzlibrary{patterns}
\usepackage[english]{babel}
\usepackage[utf8]{inputenc}
\renewcommand{\qedsymbol}{$\blacksquare$}
\renewcommand{\thesection}{$\diamond$}
\definecolor{darkgreen}{rgb}{0.0, 0.5, 0.0}
\pagestyle{fancy}
\fancyhf{}
\renewcommand{\headrulewidth}{0.4pt}
\fancyhead[R]{\thepage}
\fancyhead[L]{Linear Algebra}
\title{Linear Algebra}
\author{Report 1\\ \\ Yousef A. Abood\\ \\ ID: 900248250}
\date{June 2025}
\setlength{\parindent}{0pt}
\singlespacing
\parskip=1mm
\onehalfspacing
\begin{document}
\maketitle
\noindent\makebox[\linewidth]{\rule{15cm}{0.4pt}}
\section{Upper triangular matrix}
A matrix is called upper triangular iff every entry below the main diagonal is zero. So if a matrix $K=[k_{ij}]$ is upper triangular, then every entry $k_{ij}$ is zero when $i>j$.
\begin{multicols}{2}
\subsubsection*{Example}
$\begin{bmatrix}
  1 & 2 & 3 \\
  0 & 4 & 5 \\
  0 & 0 & 6
\end{bmatrix},$

\subsubsection*{Non-example}
$\begin{bmatrix}
  1 & 2 & 3 \\
  7 & 4 & 5 \\
  8 & 9 & 6
\end{bmatrix}.$
\end{multicols}
\section{Scalar multiplication of matrices
}
let $A=[a_{ij}]$ be a matrix of size $m\times n$, where $m,n \in \mathbb{Z}$ and let $c$ be a real number. The Scalar multiple of $A$ by $c$ is the $m \times n$ matrix $cA$ obtained by multiplying every entry of the matrix $A$ by $c$.
\subsubsection*{Example}
$ A=\begin{bmatrix}
  1 & 0 & 0 \\
  0 & 1 & 0 \\
  0 & 0 & 1
\end{bmatrix},
5A= \begin{bmatrix}
  5 & 0 & 0 \\
  0 & 5 & 0 \\
  0 & 0 & 5
\end{bmatrix}.$\\ \\
\textcolor{red}{* Non-example cannot be found.}
\section{Skew-symmetric matrix}
A square matrix $A=[a_{ij}]$ is called \textit{skew-symmetric matrix} iff $A=-A^{T}$. In other words $a_{ij}=-a_{ji}$ for every $1 \leq i,j \leq n$, which forces the main diagonal entries to be zeros.
\begin{multicols}{2}
\subsubsection*{Example}
$\begin{bmatrix}
  0 & 5 & 4 \\
  -5 & 0 & 3 \\
  -4 & -3 & 0
\end{bmatrix},$
\subsubsection*{Non-example}
$\begin{bmatrix}
  1 & 2 & 3 \\
  2 & 4 & 5 \\
  3 & 5 & 6
\end{bmatrix}.$
\end{multicols}
\section{A solution of a linear equation in 3 variables 
}
A solution of a linear equation in 3 variables is a sequence of three real numbers $(s_1,s_2,s_3)$ which satisfy the equation when the number $s_i$ is substituted for the variable $x_i$, for every $1 \leq i \leq 3$.
  \subsubsection*{Example}
The sequence $(1, 1, 1)$ is a solution for the system of equations : $x+y+z=3.$\\
\textcolor{red}{The sequence $(0,0,0)$ is a Non-example as it does not satisfy the equation.}
\section{Row-equivalent matrices}
A matrix $A$ is called $row \; equivalent$ to a matrix $B$ if and only if $B$ is obtained from $A$ by applying finitely many elementary row operations on $A$.
\subsubsection*{Example}
$
  A= \begin{bmatrix}
    7 & 12 & 5 \\
    8 & 14 & 7 \\
    1 & 9 & 3
  \end{bmatrix} \xrightarrow{{(R_1-R_3)}\to{R_1}} \begin{bmatrix}
    6 & 3 & 2 \\
    8 & 14 & 7 \\
    1 & 9 & 3
  \end{bmatrix} = B.
$
\\ \\ $A$ and $B$ are row equivalent.
\subsubsection*{Non-example}
$
A= \begin{bmatrix}
    7 & 12 & 5 \\
    8 & 14 & 7 \\
    1 & 9 & 3
  \end{bmatrix}. B=\begin{bmatrix}
    0 & 0 & 0 \\  
    0 & 0 & 0 \\
    0 & 0 & 0
  \end{bmatrix}.
$ 
\\ \\ $A$ and $B$ are not row equivalent.
\section{A matrix in row-echelon form}
A matrix $A$ is in row-echelon form iff it satisfies the three following conditions :
\begin{itemize}
  \item [(i)] All rows consisting entirely of zeros occur at the bottom of the matrix.
  \item [(ii)] The first nonzero entry of each row is 1. (called the leading 1 or the pivot, and we call the column that has a leading 1, a pivot column.)
  \item [(iii)] Any leading 1 is farther to the right than the leading 1 in the row above.
\end{itemize}

\begin{multicols}{2}
  \subsubsection*{Example}
  $
  A=\begin{bmatrix}
    1 & 5 & 8 \\
    0 & 1 & 6 \\
    0 & 0 & 1 \\
    0 & 0 & 0
  \end{bmatrix}
  $
  \subsubsection*{Non-example}
  $B=\begin{bmatrix}
    1 & 5 & 8 \\
    0 & 1 & 6 \\
    0 & 1 & 1 \\
    0 & 0 & 0
  \end{bmatrix}$
\end{multicols}

\begin{multicols}{2}
  $A$ is in row-echelon form. \\
  $B$ is not in row-echelon form.
\end{multicols}
\end{document}