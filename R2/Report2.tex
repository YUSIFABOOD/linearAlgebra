\documentclass[a4paper,12pt]{article}
\usepackage{graphicx, setspace, array, xcolor
, mathtools, contour, tikz, amsthm, setspace, amsmath,amsfonts,amssymb, subcaption, fancyhdr, lipsum,multicol, amsthm}
\usepackage[a4paper, left=2.5cm, right=2.5cm, top=3cm, bottom=2cm]{geometry}
\usepackage{colortbl}
\contourlength{0.5pt}
\usetikzlibrary{patterns}
\usepackage[english]{babel}
\usepackage[utf8]{inputenc}
\renewcommand{\qedsymbol}{$\blacksquare$}
\renewcommand{\thesection}{$\diamond$}
\definecolor{darkgreen}{rgb}{0.0, 0.5, 0.0}
\pagestyle{fancy}
\fancyhf{}
\renewcommand{\headrulewidth}{0.4pt}
\fancyhead[R]{\thepage}
\fancyhead[L]{Linear Algebra}
\title{Linear Algebra}
\author{Report 2\\ \\ Yousef A. Abood\\ \\ ID: 900248250}
\date{June 2025}
\setlength{\parindent}{0pt}
\singlespacing
\parskip=1mm
\onehalfspacing
\begin{document}
\maketitle
\noindent\makebox[\linewidth]{\rule{15cm}{0.4pt}}
\section{Invertible matrix}
A square matrix $A$ of size $n \times n$ is called invertible iff we can find another matrix $B$ such that $AB=I_{n} \text{ and } BA=I_{n}.$ We call $B$ the inverse of $A$ and we denote it by $A^{-1}.$
\\ \textcolor{red}{*When a matrix is row equivalent to a matrix with row or column of zeros then it is noninvertible matrix.}
\subsubsection*{Example}
$\left [ \begin{array}{ccc}
        1&2&5\\
        5&6&10\\
        11&12&22
    \end{array}\right ]\left [ \begin{array}{ccc}
        0&2&-2\\
        -1&\frac{7}{2}&\frac{-3}{2}\\
        \frac{1}{3}&\frac{-5}{3}&\frac{2}{3}
    \end{array}\right ]=\left [ \begin{array}{ccc}
        1&0&0\\
        0&1&0\\
        0&0&1
    \end{array}\right ] \\ \\ AB=I_{n}.$
    \subsubsection*{Non-example}
    $A=\left [ \begin{array}{ccc}
        1&2&3\\
        5&6&7\\
        10&12&14
    \end{array}\right ] \xrightarrow{{(R_3-2R_2)}\to{R_3}}\left [ \begin{array}{ccc}
        1&2&3\\
        5&6&7\\
        0&0&0
    \end{array}\right ].$\\
    \textcolor{red}{$A$ is noninvertible matrix.}
\section{Elementry matrix}
A square matrix $A$ of size $n \times n$ is called elementary iff it can be obtained from the identity matirx by applying exactly one elementary row operation.
    \subsubsection*{Example}
        $\left [ \begin{array}{ccc}
        1&0&0\\
        0&1&0\\
        0&0&1
    \end{array}\right ] \xrightarrow{{(R_3-5R_2)}	\to{R_3}} \left [ \begin{array}{ccc}
        1&0&0\\
        0&1&0\\
        0&-5&1
    \end{array}\right ] =A,$ an elementary matrix.

\subsubsection*{Example}
        $\left [ \begin{array}{ccc}
        1&0&0\\
        0&1&0\\
        0&0&1
    \end{array}\right ] \xrightarrow{{(R_3-5R_2)}	\to{R_3}} \left [ \begin{array}{ccc}
        1&0&0\\
        0&1&0\\
        0&-5&1
    \end{array}\right ] \xrightarrow{{R_2\leftrightarrow R_1}} \left [ \begin{array}{ccc}
        0&1&0\\
        1&0&0\\
        0&-5&1
    \end{array}\right ]= B,$ not an elementary matrix.
\section{Determinant of a matrix}
The determinant is a function that assaigns to every square matrix $A$ a real number \\denoted by $det(A)$ or $|A|$. For a $1 \times 1$ matrix, we define $det([a])=a.$ The determinant of an $n\times n$ matrix, where $n \geq 2$, is the sum of the products of the entries of the first row with their corresponding cofactors.\\
\[det(A)=\sum_{j=1}^{n}(-1)^{1+j}a_{1j}det(A_{1j})=\sum_{j=1}^{n}a_{1j}C_{1j}=a_{11}C_{11}+a_{12}C_{12}+...+a_{1n}C_{1n}.\]
\subsubsection*{Example}
\[
A=\left [\begin{array}{ccc}
    9&11&13\\
    17&19&23\\
    29&31&37
\end{array} \right ].\] \\ \[det(A) =\left |\begin{array}{ccc}
    9&11&13\\
    17&19&23\\
    29&31&37
\end{array} \right |= 9 \left |\begin{array}{ccc}
       19 &23\\
        31&37
\end{array}\right |- 11 \left |\begin{array}{ccc}
       17 &23\\
        29&37
\end{array}\right |+ 13 \left |\begin{array}{lll}
       17 &19\\
        29&31
\end{array}\right |=-90+418-312=16.
\]
\textcolor{red}{* A nonexample cannot be found.}
\end{document}