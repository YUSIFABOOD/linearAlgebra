\documentclass[12pt]{article}
\usepackage{graphicx} % Required for inserting images
\usepackage{amsmath}
\usepackage{listings}
\usepackage{geometry}
\usepackage[most]{tcolorbox}
\usepackage{graphicx, setspace, array, xcolor, mathtools, contour, tikz, amsthm, setspace, amsmath,amsfonts,amssymb, subcaption, fancyhdr, lipsum,multicol}
\geometry{a4paper, margin=1in}
% ===== Define custom colors =====
\definecolor{defscol}{HTML}{d8f5d1}

% ===== Define theorem-like environment for definitions =====
\newtcbtheorem[]{mydefinition}{Definition}
{
    enhanced,
    frame hidden,
    titlerule=0mm,
    toptitle=1mm,
    bottomtitle=1mm,
    fonttitle=\bfseries\large,
    coltitle=black,
     colbacktitle=defscol!60!white,  % darker green for title
    colback=defscol!30!white,       % lighter green for body
    % width=0.8\textwidth,            % 80% of the page width
    % halign=center,                  % center the entire box
    boxsep=5pt        
}{defn}

% ===== Corrected Shortcut Commands =====

% A command for a standard definition.
% Usage: \defn{Title}{label-key}{Content...}
\NewDocumentCommand{\defn}{m m +m}{
    \begin{mydefinition}{#1}{#2}
        #3
    \end{mydefinition}
}

% A command for a definition with an extra note in the title.
% The note is placed in parentheses after the title.
% Usage: \defnr{Title}{Note}{label-key}{Content...}
\NewDocumentCommand{\defnr}{m m m +m}{
    \begin{mydefinition}{#1 (#2)}{#3}
        #4
    \end{mydefinition}
}
% =======================================================
% Code for Theorems with Proofs
% =======================================================

% Required Packages (make sure these are in your preamble)
\usepackage{xparse}

% --- 1. Define a color for theorems ---
\definecolor{thmscol}{HTML}{d7e6ec} % A light blue color

% --- 2. Define the theorem environment style ---
\newtcbtheorem[number within=section]{mytheorem}{Theorem}
{
    enhanced,
    frame hidden,
    titlerule=0mm,
    toptitle=1mm,
    bottomtitle=1mm,
    fonttitle=\bfseries\large,
    coltitle=black,
    colbacktitle=thmscol!40!white,
    colback=thmscol!20!white,
}{thm}
% Command for a simple theorem
\NewDocumentCommand{\thm}{m+m}{
    \begin{mytheorem}{#1}{}
        #2
    \end{mytheorem}
}
% --- 3. Define the proof environment style ---
\newenvironment{thmpf}{
    {\noindent\textbf{Proof.}}
    \tcolorbox[blanker, breakable, left=5mm, parbox=false,
    before upper={\parindent15pt},
    after skip=10pt,
    borderline west={1mm}{0pt}{thmscol!60!white}]
}{
    % This adds the black square at the end of the proof
    \textcolor{thmscol!60!black}{\hbox{}\nobreak\hfill$\blacksquare$}
    \endtcolorbox
}

% --- 4. Create the all-in-one command for a theorem with its proof ---
% Usage: \thmp{Title}{label}{Theorem Content}{Proof Content}
\NewDocumentCommand{\thmp}{m m +m +m}{
    \begin{mytheorem}{#1}{#2}
        #3
    \end{mytheorem}
    \begin{thmpf}
        #4
    \end{thmpf}
}
\renewcommand{\thesection}{$\diamond$}
\definecolor{darkgreen}{rgb}{0.0, 0.5, 0.0}
\pagestyle{fancy}
\fancyhf{}
\renewcommand{\headrulewidth}{0.4pt}
\fancyhead[R]{\thepage}
\fancyhead[L]{Linear Algebra}
\title{Linear Algebra}
\author{Report 5\\ \\ Yousef A. Abood\\ \\ ID: 900248250}
\date{Jully 2025}
\setlength{\parindent}{0pt}
\singlespacing
\parskip=1mm
\onehalfspacing
\begin{document}
\maketitle
\noindent\makebox[\linewidth]{\rule{15cm}{0.4pt}}
\defn{Linear transformation}{lt}{
Let $W,V$ be two vector spaces and $T$ is a function that its domain is $V$ and its codomain is $W$. We call $T$ a linear transformation iff for any $c \in \mathbb{R}$ and for every vector $\vec{v}, \vec{u} \in V$ the following conditions hold:
    \begin{itemize}
        \item $T(\vec{v}+\vec{u})= T(\vec{v})+T(\vec{u})$
        \item $T(c\vec{v})=cT(\vec{v}).$
    \end{itemize}
} 

\subsubsection*{Example}
The linear transformation $T:V \to V$ such that $T(x,y)=(x+y,3x-7y).$ We see it is a linear transformation as it is closed under vector addition and scalar multiplication as follows
Pick two vectors $\vec{u}=(x,y), \vec{v}=(a,b) \in V$ and scalar $c \in \mathbb{R}$, we see
\begin{align*}
    T(\vec{u}+\vec{v})&=T((x+a,y+b))\\
    &=(x+a+y+b, 3x+3a-7y-7b)\\
     &=(x+y,3x-7y)+(a+b,3a-7b)=T(\vec{u})+T(\vec{v})\\
\\
    T(c\vec{v})&=T((ca,cb))=(ca+cb, 3ca-7cb) \\ 
    &= (c(a+b), c(3a-7b))=c(a+b, 3a-7b)=cT(\vec{v}).
\end{align*}

\subsubsection*{\textcolor{red}{Non-example}}
The function $T(x,y)=(x+1,y-1)$ is not a linear transformation as it is not closed under vector addition. We pick two vectors $\vec{v}=(x,y), \vec{u}=(a,b) \in V$ and see
\[T(\vec{v}+ \vec{u})= T(a+x,b+y)=(a+b+1, b+y-1)\]
Which clearly does not satisfy the closure of addition.
\defn{Kernel of a linear transformation}{ker}{
Let $T: V \to W$ be a linear transformation. We define the kernel of the linear transformation as the set of all vectors in $V$ that are mapped to the zero vector of $W$. that is, \[kernel(T)=\{\vec{v}\in V| T(\vec{v})=\vec{\textbf{0}}_W\}\]
}
\subsubsection*{Example}
Let $T: \mathbb{R}^3\to \mathbb{R}^3$ be the projection of a vector into the $yz-plane$, such that $T(x,y,z)=(0,y,z).$
Then $ker(T) = \{(t,0,0)| t \in \mathbb{R}.\}$ 

\subsubsection*{\textcolor{red}{Non-example}}
A non example is any vector in $V$ that its image is not $\vec{\textbf{0}}_W$. An example is $S=\{(1,1,1)\}$, where $T(1,1,1)=(0,1,1).$
\defn{Vector space isomorphism}{iso}{
Let $V,W$ be two vector spaces. An \textit{isomorphism} from $V$ to $W$ is a bijective linear transformation $T: V \to W.$ We say $V$ is isomorphic to $W$ iff we can find at least one isomorphism from $V$ to $W$ and we write that $V \cong W.$
}

\subsubsection*{Example}
The vector spaces
\[R^5\cong M_{5\times 1} \cong M_{1 \times 5} \cong P_4\]
are all isomorphic to each other since they all are of dimension 5.

\subsubsection*{\textcolor{red}{Non-example}}
The vector spaces
\[R^3 \ncong M_{2 \times 2}\]
are not isomorphic since they are of different dimensions, the first of dimension 3 and the latter of dimension 4.
\defn{Eigenvector of a matrix}{eigen}{
A nonzero vector $\vec{x} \in R^n$ is called an eigenvector of a square matrix $A$ of size
$n \times n$ if and only if there exists a real number $\lambda \in R$ such that
\[A \vec{x} = \lambda \vec{x}\].
The number $\lambda$ is called an eigenvalue of $A$, and we say in such situation that
$\vec{x}$ is an eigenvector of $A$ corresponding to the eigenvalue $\lambda$.
}
\subsubsection*{Example}
let $A$ be the $2 \times 2$ matrix
\[A=\begin{bmatrix}
  1 & 4 \\
  5 & 0
\end{bmatrix}\]
An eigenvector is 
\[\vec{v}=\begin{bmatrix}
    1\\
    1
\end{bmatrix}\]
As we can see that 
\[A\vec{v}=\begin{bmatrix}
  1 & 4 \\
  5 & 0
\end{bmatrix}\begin{bmatrix}
    1\\
    1
\end{bmatrix}=\begin{bmatrix}
    5\\
    5
\end{bmatrix}=5\begin{bmatrix}
    1\\
    1
\end{bmatrix}=5 \vec{v}.\]
\subsubsection*{\textcolor{red}{Non-example}}
For the previous matrix $A$ let \[\vec{v}=\begin{bmatrix}
    1\\
    2
\end{bmatrix}\]
And observe
\[A\vec{v}=\begin{bmatrix}
  1 & 4 \\
  5 & 0
\end{bmatrix}\begin{bmatrix}
    1\\
    2
\end{bmatrix}=\begin{bmatrix}
    9\\
    5
\end{bmatrix},\]
which is not an eigenvector as there is no $\lambda \in \mathbb{R}$ such that $A\vec{v}=\lambda \vec{v.}$
\end{document}